% generated from JIRA project LVV
% using template at <template>.
% Collecting ATM data from folder: "/Data Management/Software Products/Supporting SW/Distrib Database|LDM-552"
% using docsteady version 1.2rc2.post55+g044fd29
% Please do not edit -- update information in Jira instead

\section{Test Cases Summary}\label{test-cases-summary}

\begin{longtable}[]{p{3cm}p{13cm}}
\toprule
Test Id & Test Name\tabularnewline
\midrule
\endhead
    \hyperref[lvv-t1017]{LVV-T1017} &
    \href{https://jira.lsstcorp.org/secure/Tests.jspa\#/testCase/LVV-T1017}{Qserv Preparation
} \tabularnewline
    \hyperref[lvv-t1085]{LVV-T1085} &
    \href{https://jira.lsstcorp.org/secure/Tests.jspa\#/testCase/LVV-T1085}{Short Queries Functional Test
} \tabularnewline
    \hyperref[lvv-t1086]{LVV-T1086} &
    \href{https://jira.lsstcorp.org/secure/Tests.jspa\#/testCase/LVV-T1086}{Full Table Scans Functional Test
} \tabularnewline
    \hyperref[lvv-t1087]{LVV-T1087} &
    \href{https://jira.lsstcorp.org/secure/Tests.jspa\#/testCase/LVV-T1087}{Full Table Joins Functional Test
} \tabularnewline
    \hyperref[lvv-t1088]{LVV-T1088} &
    \href{https://jira.lsstcorp.org/secure/Tests.jspa\#/testCase/LVV-T1088}{Concurrent Scans Scaling Test
} \tabularnewline
    \hyperref[lvv-t1089]{LVV-T1089} &
    \href{https://jira.lsstcorp.org/secure/Tests.jspa\#/testCase/LVV-T1089}{Load Test
} \tabularnewline
    \hyperref[lvv-t1090]{LVV-T1090} &
    \href{https://jira.lsstcorp.org/secure/Tests.jspa\#/testCase/LVV-T1090}{Heavy Load Test
} \tabularnewline
\bottomrule
\end{longtable}

\newpage

\section{Active Test Cases}

This section documents all active test cases that have a status in the Jira/ATM system of Draft, Defined or Approved.



\subsection{\href{https://jira.lsstcorp.org/secure/Tests.jspa\#/testCase/LVV-T1017}{LVV-T1017}
    - Qserv Preparation
}\label{lvv-t1017}

\begin{longtable}[]{llllll}
\toprule
Version & Status & Priority & Verification Type & Owner
\\\midrule
1 & Approved & Normal &
Test & Fritz Mueller
\\\bottomrule
\end{longtable}

\subsubsection{Verification Elements}
    None.

\subsubsection{Test Items}
Before running any of the performance test cases, Qserv must be
installed on an appropriate test cluster (e.g. the test machine cluster
at CC-IN2P3). ~To upgrade Qserv software on the cluster in preparation
for testing, follow directions at
http://www.slac.stanford.edu/exp/lsst/qserv/2015\_10/HOW-TO/cluster-deployment.html.\\[2\baselineskip]The
performance tests will also require an appropriately sized test dataset
to be synthesized and ingested, per the yearly dataset sizing schedule
described in \citeds{LDM-552}, section 2.2.1. ~Tools for synthesis of ingest of
test datasets may be found in the LSST GitHub repot at
https://github.com/lsst-dm/db\_tests\_kpm*. ~Detailed use and context
information for the tools is described in
https://jira.lsstcorp.org/browse/DM-8405.\\[2\baselineskip]It has also
been found that the Qserv shard servers must have engine-independent
statistics loaded for the larger tables in the test dataset, and be
properly configured so that the MariaDB query planner can make use of
those statistics. ~More information on this issue is available at
https://confluence.lsstcorp.org/pages/viewpage.action?pageId=58950786.



\subsubsection{Predecessors}

\subsubsection{Environment Needs}

\paragraph{Software}

\paragraph{Hardware}

\subsubsection{Input Specification}

\subsubsection{Output Specification}

\subsubsection{Test Procedure}
    \begin{longtable}[]{p{1.3cm}p{2cm}p{13cm}}
    %\toprule
    Step & \multicolumn{2}{@{}l}{Description, Input Data and Expected Result} \\ \toprule
    \endhead

            \multirow{3}{*}{ 1 } & Description &
            \begin{minipage}[t]{13cm}{\footnotesize
            Install/upgrade Qserv on a test cluster, following directions at
http://www.slac.stanford.edu/exp/lsst/qserv/2015\_10/HOW-TO/cluster-deployment.html

            \vspace{\dp0}
            } \end{minipage} \\ \cline{2-3}
            & Test Data &
            \begin{minipage}[t]{13cm}{\footnotesize
                No data.
                \vspace{\dp0}
            } \end{minipage} \\ \cline{2-3}
            & Expected Result &
                \begin{minipage}[t]{13cm}{\footnotesize
                Qserv installed

                \vspace{\dp0}
                } \end{minipage}
        \\ \midrule

            \multirow{3}{*}{ 2 } & Description &
            \begin{minipage}[t]{13cm}{\footnotesize
            Synthesize and load and appropriately sized test dataset per the yearly
dataset sizing schedule described in LDM-552, section 2.2.1. Tools for
synthesis of ingest of test datasets may be found in the LSST GitHub
repot at https://github.com/lsst dm/db\_tests\_kpm*. ~Detailed use and
context information for the tools is described in
https://jira.lsstcorp.org/browse/DM-8405.\\[2\baselineskip]

            \vspace{\dp0}
            } \end{minipage} \\ \cline{2-3}
            & Test Data &
            \begin{minipage}[t]{13cm}{\footnotesize
                No data.
                \vspace{\dp0}
            } \end{minipage} \\ \cline{2-3}
            & Expected Result &
                \begin{minipage}[t]{13cm}{\footnotesize
                Test dataset loaded

                \vspace{\dp0}
                } \end{minipage}
        \\ \midrule
    \end{longtable}

\subsection{\href{https://jira.lsstcorp.org/secure/Tests.jspa\#/testCase/LVV-T1085}{LVV-T1085}
    - Short Queries Functional Test
}\label{lvv-t1085}

\begin{longtable}[]{llllll}
\toprule
Version & Status & Priority & Verification Type & Owner
\\\midrule
1 & Approved & Normal &
Test & Fritz Mueller
\\\bottomrule
\end{longtable}

\subsubsection{Verification Elements}
\begin{itemize}
\item \href{https://jira.lsstcorp.org/browse/LVV-33}{LVV-33} - DMS-REQ-0075-V-01: Catalog Queries

\item \href{https://jira.lsstcorp.org/browse/LVV-9787}{LVV-9787} - DMS-REQ-0356-V-04: Max time to retrieve low-volume query results

\end{itemize}

\subsubsection{Test Items}
The objective of this test is to ensure that the short queries are
performing as expected and establish a timing baseline benchmark for
these types of queries.



\subsubsection{Predecessors}

\subsubsection{Environment Needs}

\paragraph{Software}

\paragraph{Hardware}

\subsubsection{Input Specification}
QSERV has been set-up following procedure at ~LVV-T1017.


\subsubsection{Output Specification}

\subsubsection{Test Procedure}
    \begin{longtable}[]{p{1.3cm}p{2cm}p{13cm}}
    %\toprule
    Step & \multicolumn{2}{@{}l}{Description, Input Data and Expected Result} \\ \toprule
    \endhead

            \multirow{3}{*}{ 1 } & Description &
            \begin{minipage}[t]{13cm}{\footnotesize
            Execute single object selection:\\[2\baselineskip]\textbf{SELECT} *
\textbf{FROM} Object~\textbf{WHERE} deepSourceId =
9292041530376264\\[2\baselineskip]and record execution time.

            \vspace{\dp0}
            } \end{minipage} \\ \cline{2-3}
            & Test Data &
            \begin{minipage}[t]{13cm}{\footnotesize
                No data.
                \vspace{\dp0}
            } \end{minipage} \\ \cline{2-3}
            & Expected Result &
                \begin{minipage}[t]{13cm}{\footnotesize
                Query runs in less than 10 seconds.

                \vspace{\dp0}
                } \end{minipage}
        \\ \midrule

            \multirow{3}{*}{ 2 } & Description &
            \begin{minipage}[t]{13cm}{\footnotesize
            Execute spatial area selection from
Object:\\[2\baselineskip]\textbf{SELECT COUNT(*)} \textbf{FROM} Object
\textbf{WHERE}~\\

~qserv\_areaspec\_box(316.582327, −6.839078, 316.653938, −6.781822)

and record execution time.

            \vspace{\dp0}
            } \end{minipage} \\ \cline{2-3}
            & Test Data &
            \begin{minipage}[t]{13cm}{\footnotesize
                No data.
                \vspace{\dp0}
            } \end{minipage} \\ \cline{2-3}
            & Expected Result &
                \begin{minipage}[t]{13cm}{\footnotesize
                Query runs in less than 10 seconds.

                \vspace{\dp0}
                } \end{minipage}
        \\ \midrule
    \end{longtable}

\subsection{\href{https://jira.lsstcorp.org/secure/Tests.jspa\#/testCase/LVV-T1086}{LVV-T1086}
    - Full Table Scans Functional Test
}\label{lvv-t1086}

\begin{longtable}[]{llllll}
\toprule
Version & Status & Priority & Verification Type & Owner
\\\midrule
1 & Approved & Normal &
Test & Fritz Mueller
\\\bottomrule
\end{longtable}

\subsubsection{Verification Elements}
\begin{itemize}
\item \href{https://jira.lsstcorp.org/browse/LVV-33}{LVV-33} - DMS-REQ-0075-V-01: Catalog Queries

\item \href{https://jira.lsstcorp.org/browse/LVV-188}{LVV-188} - DMS-REQ-0357-V-01: Result latency for high-volume full-sky queries on
the Object table

\item \href{https://jira.lsstcorp.org/browse/LVV-185}{LVV-185} - DMS-REQ-0354-V-01: Result latency for high-volume complex queries

\end{itemize}

\subsubsection{Test Items}
The objective of this test is to ensure that the full table scan queries
are performing as expected and establish a timing baseline benchmark for
these types of queries.



\subsubsection{Predecessors}

\subsubsection{Environment Needs}

\paragraph{Software}

\paragraph{Hardware}

\subsubsection{Input Specification}
QSERV has been set-up following procedure at ~LVV-T1017.


\subsubsection{Output Specification}

\subsubsection{Test Procedure}
    \begin{longtable}[]{p{1.3cm}p{2cm}p{13cm}}
    %\toprule
    Step & \multicolumn{2}{@{}l}{Description, Input Data and Expected Result} \\ \toprule
    \endhead

            \multirow{3}{*}{ 1 } & Description &
            \begin{minipage}[t]{13cm}{\footnotesize
            Execute query:\\[2\baselineskip]\textbf{SELECT} ra , decl , u\_psfFlux ,
g\_psfFlux , r\_psfFlux \textbf{FROM} Object\\
\textbf{WHERE} y\_shapeIxx \textbf{BETWEEN} 20 \textbf{AND}
20.1\\[3\baselineskip]and record execution time and output size.

            \vspace{\dp0}
            } \end{minipage} \\ \cline{2-3}
            & Test Data &
            \begin{minipage}[t]{13cm}{\footnotesize
                No data.
                \vspace{\dp0}
            } \end{minipage} \\ \cline{2-3}
            & Expected Result &
                \begin{minipage}[t]{13cm}{\footnotesize
                Query expected to run in less than 1 hour.\\[2\baselineskip]

                \vspace{\dp0}
                } \end{minipage}
        \\ \midrule

            \multirow{3}{*}{ 2 } & Description &
            \begin{minipage}[t]{13cm}{\footnotesize
            Execute query:\\[2\baselineskip]\textbf{SELECT} COUNT(*) \textbf{FROM}
Source \textbf{WHERE} flux\_sinc \textbf{BETWEEN} 1 \textbf{AND}
1.1\\[2\baselineskip]and record the execution time

            \vspace{\dp0}
            } \end{minipage} \\ \cline{2-3}
            & Test Data &
            \begin{minipage}[t]{13cm}{\footnotesize
                No data.
                \vspace{\dp0}
            } \end{minipage} \\ \cline{2-3}
            & Expected Result &
                \begin{minipage}[t]{13cm}{\footnotesize
                Query expected to run in less than 12 hours.

                \vspace{\dp0}
                } \end{minipage}
        \\ \midrule

            \multirow{3}{*}{ 3 } & Description &
            \begin{minipage}[t]{13cm}{\footnotesize
            Execute query:\\[2\baselineskip]\textbf{SELECT} COUNT(*) \textbf{FROM}
ForcedSource \textbf{WHERE} psfFlux \textbf{BETWEEN} 0.1 \textbf{AND}
0.2\\[2\baselineskip]and record the execution time

            \vspace{\dp0}
            } \end{minipage} \\ \cline{2-3}
            & Test Data &
            \begin{minipage}[t]{13cm}{\footnotesize
                No data.
                \vspace{\dp0}
            } \end{minipage} \\ \cline{2-3}
            & Expected Result &
                \begin{minipage}[t]{13cm}{\footnotesize
                Query expected to run in less than 12 hours.

                \vspace{\dp0}
                } \end{minipage}
        \\ \midrule
    \end{longtable}

\subsection{\href{https://jira.lsstcorp.org/secure/Tests.jspa\#/testCase/LVV-T1087}{LVV-T1087}
    - Full Table Joins Functional Test
}\label{lvv-t1087}

\begin{longtable}[]{llllll}
\toprule
Version & Status & Priority & Verification Type & Owner
\\\midrule
1 & Approved & Normal &
Test & Fritz Mueller
\\\bottomrule
\end{longtable}

\subsubsection{Verification Elements}
\begin{itemize}
\item \href{https://jira.lsstcorp.org/browse/LVV-33}{LVV-33} - DMS-REQ-0075-V-01: Catalog Queries

\item \href{https://jira.lsstcorp.org/browse/LVV-185}{LVV-185} - DMS-REQ-0354-V-01: Result latency for high-volume complex queries

\end{itemize}

\subsubsection{Test Items}
The objective of this test is to ensure that the full table join queries
are performing as expected and establish a timing baseline benchmark for
these types of queries.



\subsubsection{Predecessors}

\subsubsection{Environment Needs}

\paragraph{Software}

\paragraph{Hardware}

\subsubsection{Input Specification}
QSERV has been set-up following procedure at ~LVV-T1017.


\subsubsection{Output Specification}

\subsubsection{Test Procedure}
    \begin{longtable}[]{p{1.3cm}p{2cm}p{13cm}}
    %\toprule
    Step & \multicolumn{2}{@{}l}{Description, Input Data and Expected Result} \\ \toprule
    \endhead

            \multirow{3}{*}{ 1 } & Description &
            \begin{minipage}[t]{13cm}{\footnotesize
            Execute query:\\[2\baselineskip]\textbf{SELECT} o.deepSourceId,
s.objectId, s.id, o.ra, o.decl\\
\textbf{~ ~ FROM} Object o, Source s WHERE o.deepSourceId=s.objectId\\
\hspace*{0.333em} ~ \textbf{AND} s . flux\_sinc \textbf{BETWEEN} 0.3
\textbf{AND} 0.31\\[2\baselineskip]and record execution time.

            \vspace{\dp0}
            } \end{minipage} \\ \cline{2-3}
            & Test Data &
            \begin{minipage}[t]{13cm}{\footnotesize
                No data.
                \vspace{\dp0}
            } \end{minipage} \\ \cline{2-3}
            & Expected Result &
                \begin{minipage}[t]{13cm}{\footnotesize
                Query expected to run in less than 12 hours.

                \vspace{\dp0}
                } \end{minipage}
        \\ \midrule

            \multirow{3}{*}{ 2 } & Description &
            \begin{minipage}[t]{13cm}{\footnotesize
            Execute query:\\[2\baselineskip]\textbf{SELECT} o.deepSourceId,
f.psfFlux \textbf{FROM} Object o, ForcedSource f\\
\textbf{~ ~ WHERE} o.deepSourceId=f.deepSourceId\\
\textbf{~ ~ AND} f . psfFlux \textbf{BETWEEN} 0.13 \textbf{AND}
0.14\\[2\baselineskip]and record execution time.

            \vspace{\dp0}
            } \end{minipage} \\ \cline{2-3}
            & Test Data &
            \begin{minipage}[t]{13cm}{\footnotesize
                No data.
                \vspace{\dp0}
            } \end{minipage} \\ \cline{2-3}
            & Expected Result &
                \begin{minipage}[t]{13cm}{\footnotesize
                Query expected to run in less than 12 hours.

                \vspace{\dp0}
                } \end{minipage}
        \\ \midrule
    \end{longtable}

\subsection{\href{https://jira.lsstcorp.org/secure/Tests.jspa\#/testCase/LVV-T1088}{LVV-T1088}
    - Concurrent Scans Scaling Test
}\label{lvv-t1088}

\begin{longtable}[]{llllll}
\toprule
Version & Status & Priority & Verification Type & Owner
\\\midrule
1 & Approved & Normal &
Test & Fritz Mueller
\\\bottomrule
\end{longtable}

\subsubsection{Verification Elements}
\begin{itemize}
\item \href{https://jira.lsstcorp.org/browse/LVV-185}{LVV-185} - DMS-REQ-0354-V-01: Result latency for high-volume complex queries

\item \href{https://jira.lsstcorp.org/browse/LVV-188}{LVV-188} - DMS-REQ-0357-V-01: Result latency for high-volume full-sky queries on
the Object table

\item \href{https://jira.lsstcorp.org/browse/LVV-3403}{LVV-3403} - DMS-REQ-0361-V-01: Simultaneous users for high-volume queries

\end{itemize}

\subsubsection{Test Items}
This test will show that average completion-time of full-scan queries of
the Object catalog table grows sub-linearly with respect to the number
of simultaneously active full-scan queries, within the limits of machine
resource exhaustion.



\subsubsection{Predecessors}

\subsubsection{Environment Needs}

\paragraph{Software}

\paragraph{Hardware}

\subsubsection{Input Specification}
\begin{enumerate}
\tightlist
\item
  A test catalog of appropriate size (see schedule detail in \citeds{LDM-552},
  section 2.2.1), prepared and ingested into the Qserv instance under
  test as detailed in LVV-T1017.
\item
  The concurrency load execution script, runQueries.py, maintained in
  the LSST Qserv github repository here:
  https://github.com/lsst/qserv/blob/master/admin/tools/docker/deployment/in2p3/runQueries.py
\end{enumerate}


\subsubsection{Output Specification}

\subsubsection{Test Procedure}
    \begin{longtable}[]{p{1.3cm}p{2cm}p{13cm}}
    %\toprule
    Step & \multicolumn{2}{@{}l}{Description, Input Data and Expected Result} \\ \toprule
    \endhead

            \multirow{3}{*}{ 1 } & Description &
            \begin{minipage}[t]{13cm}{\footnotesize
            Repeat steps 2 through 5 below, where ``pool of interest'' is taken
first to be ``FTSObj'' and subsequently ``FTSSrc'':

            \vspace{\dp0}
            } \end{minipage} \\ \cline{2-3}
            & Test Data &
            \begin{minipage}[t]{13cm}{\footnotesize
                No data.
                \vspace{\dp0}
            } \end{minipage} \\ \cline{2-3}
            & Expected Result &
                \begin{minipage}[t]{13cm}{\footnotesize
                At end of each pass, a graph indicating scan scaling rate and machine
resource exhaustion cutoff.

                \vspace{\dp0}
                } \end{minipage}
        \\ \midrule

            \multirow{3}{*}{ 2 } & Description &
            \begin{minipage}[t]{13cm}{\footnotesize
            Inspect and modify the CONCURRENCY and TARGET\_RATES dictionaries in the
runQueries.py script. Set CONCURRENCY initially to 1 for the query pool
of interest, and to 0 for all other query pools. Set TARGET\_RATES for
the query pool of interest to the yearly value per table in LDM-552,
section 2.2.1.

            \vspace{\dp0}
            } \end{minipage} \\ \cline{2-3}
            & Test Data &
            \begin{minipage}[t]{13cm}{\footnotesize
                No data.
                \vspace{\dp0}
            } \end{minipage} \\ \cline{2-3}
            & Expected Result &
                \begin{minipage}[t]{13cm}{\footnotesize
                rueQueries.py script updated with appropriate values for test iteration

                \vspace{\dp0}
                } \end{minipage}
        \\ \midrule

            \multirow{3}{*}{ 3 } & Description &
            \begin{minipage}[t]{13cm}{\footnotesize
            Execute the runQueries.py script and let it run for at least one, but
preferably several, query cycles.

            \vspace{\dp0}
            } \end{minipage} \\ \cline{2-3}
            & Test Data &
            \begin{minipage}[t]{13cm}{\footnotesize
                No data.
                \vspace{\dp0}
            } \end{minipage} \\ \cline{2-3}
            & Expected Result &
                \begin{minipage}[t]{13cm}{\footnotesize
                Test script executes producing log file.

                \vspace{\dp0}
                } \end{minipage}
        \\ \midrule

            \multirow{3}{*}{ 4 } & Description &
            \begin{minipage}[t]{13cm}{\footnotesize
            Examine log file output and compile performance statistics to obtain a
growth curve point for the pool of interest for the test report.

            \vspace{\dp0}
            } \end{minipage} \\ \cline{2-3}
            & Test Data &
            \begin{minipage}[t]{13cm}{\footnotesize
                No data.
                \vspace{\dp0}
            } \end{minipage} \\ \cline{2-3}
            & Expected Result &
                \begin{minipage}[t]{13cm}{\footnotesize
                Logs indicate either successful test run, providing another growth point
for curve, or errors indicating machine resource exhaustion cutoff has
been reached.

                \vspace{\dp0}
                } \end{minipage}
        \\ \midrule

            \multirow{3}{*}{ 5 } & Description &
            \begin{minipage}[t]{13cm}{\footnotesize
            Adjust the CONCURRENCY value for the pool of interest and repeat from
step 3 to establish the growth trend and machine resource exhaustion
cutoff for the query pool of interest to an acceptable degree of
accuracy.

            \vspace{\dp0}
            } \end{minipage} \\ \cline{2-3}
            & Test Data &
            \begin{minipage}[t]{13cm}{\footnotesize
                No data.
                \vspace{\dp0}
            } \end{minipage} \\ \cline{2-3}
            & Expected Result &
                \begin{minipage}[t]{13cm}{\footnotesize
                Average query execution time for full scan queries of each class should
be demonstrated to grow sub-linearly in the number of concurrent queries
to the limits of machine resource exhaustion.

                \vspace{\dp0}
                } \end{minipage}
        \\ \midrule
    \end{longtable}

\subsection{\href{https://jira.lsstcorp.org/secure/Tests.jspa\#/testCase/LVV-T1089}{LVV-T1089}
    - Load Test
}\label{lvv-t1089}

\begin{longtable}[]{llllll}
\toprule
Version & Status & Priority & Verification Type & Owner
\\\midrule
1 & Approved & Normal &
Test & Fritz Mueller
\\\bottomrule
\end{longtable}

\subsubsection{Verification Elements}
\begin{itemize}
\item \href{https://jira.lsstcorp.org/browse/LVV-9786}{LVV-9786} - DMS-REQ-0356-V-03: Min number of simultaneous low-volume query users

\item \href{https://jira.lsstcorp.org/browse/LVV-9787}{LVV-9787} - DMS-REQ-0356-V-04: Max time to retrieve low-volume query results

\item \href{https://jira.lsstcorp.org/browse/LVV-188}{LVV-188} - DMS-REQ-0357-V-01: Result latency for high-volume full-sky queries on
the Object table

\item \href{https://jira.lsstcorp.org/browse/LVV-185}{LVV-185} - DMS-REQ-0354-V-01: Result latency for high-volume complex queries

\item \href{https://jira.lsstcorp.org/browse/LVV-3403}{LVV-3403} - DMS-REQ-0361-V-01: Simultaneous users for high-volume queries

\end{itemize}

\subsubsection{Test Items}
This test will check that Qserv is able to meet average query completion
time targets per query class under a representative load of simultaneous
high and low volume queries while running against an appropriately
scaled test catalog.



\subsubsection{Predecessors}

\subsubsection{Environment Needs}

\paragraph{Software}

\paragraph{Hardware}

\subsubsection{Input Specification}
QSERV has been set-up following procedure at ~LVV-T1017


\subsubsection{Output Specification}

\subsubsection{Test Procedure}
    \begin{longtable}[]{p{1.3cm}p{2cm}p{13cm}}
    %\toprule
    Step & \multicolumn{2}{@{}l}{Description, Input Data and Expected Result} \\ \toprule
    \endhead

            \multirow{3}{*}{ 1 } & Description &
            \begin{minipage}[t]{13cm}{\footnotesize
            Inspect and modify the CONCURRENCY and TARGET\_RATES dictionaries in the
runQueries.py script. ~Set CONCURRENCY and TARGET\_RATES for all pools
to the yearly value per table in LDM-552, section 2.2.1.

            \vspace{\dp0}
            } \end{minipage} \\ \cline{2-3}
            & Test Data &
            \begin{minipage}[t]{13cm}{\footnotesize
                No data.
                \vspace{\dp0}
            } \end{minipage} \\ \cline{2-3}
            & Expected Result &
                \begin{minipage}[t]{13cm}{\footnotesize
                Script updated with appropriate values.

                \vspace{\dp0}
                } \end{minipage}
        \\ \midrule

            \multirow{3}{*}{ 2 } & Description &
            \begin{minipage}[t]{13cm}{\footnotesize
            Execute the runQueries.py script and let it run for 24 hours.

            \vspace{\dp0}
            } \end{minipage} \\ \cline{2-3}
            & Test Data &
            \begin{minipage}[t]{13cm}{\footnotesize
                No data.
                \vspace{\dp0}
            } \end{minipage} \\ \cline{2-3}
            & Expected Result &
                \begin{minipage}[t]{13cm}{\footnotesize
                Script runs without error and produces output log.

                \vspace{\dp0}
                } \end{minipage}
        \\ \midrule

            \multirow{3}{*}{ 3 } & Description &
            \begin{minipage}[t]{13cm}{\footnotesize
            Examine log file output and compile average query execution times per
query type; and compare to yearly target values per table in LDM-552,
section 2.2.1.

            \vspace{\dp0}
            } \end{minipage} \\ \cline{2-3}
            & Test Data &
            \begin{minipage}[t]{13cm}{\footnotesize
                No data.
                \vspace{\dp0}
            } \end{minipage} \\ \cline{2-3}
            & Expected Result &
                \begin{minipage}[t]{13cm}{\footnotesize
                Average query times per query type equal or less than corresponding
yearly target values in LDM-552, section 2.2.1.

                \vspace{\dp0}
                } \end{minipage}
        \\ \midrule
    \end{longtable}

\subsection{\href{https://jira.lsstcorp.org/secure/Tests.jspa\#/testCase/LVV-T1090}{LVV-T1090}
    - Heavy Load Test
}\label{lvv-t1090}

\begin{longtable}[]{llllll}
\toprule
Version & Status & Priority & Verification Type & Owner
\\\midrule
1 & Approved & Normal &
Test & Fritz Mueller
\\\bottomrule
\end{longtable}

\subsubsection{Verification Elements}
\begin{itemize}
\item \href{https://jira.lsstcorp.org/browse/LVV-9786}{LVV-9786} - DMS-REQ-0356-V-03: Min number of simultaneous low-volume query users

\item \href{https://jira.lsstcorp.org/browse/LVV-9787}{LVV-9787} - DMS-REQ-0356-V-04: Max time to retrieve low-volume query results

\item \href{https://jira.lsstcorp.org/browse/LVV-188}{LVV-188} - DMS-REQ-0357-V-01: Result latency for high-volume full-sky queries on
the Object table

\item \href{https://jira.lsstcorp.org/browse/LVV-185}{LVV-185} - DMS-REQ-0354-V-01: Result latency for high-volume complex queries

\item \href{https://jira.lsstcorp.org/browse/LVV-3403}{LVV-3403} - DMS-REQ-0361-V-01: Simultaneous users for high-volume queries

\end{itemize}

\subsubsection{Test Items}
This test will check that Qserv is able to meet average query completion
time targets per query class under a higher than average load of
simultaneous high and low volume queries while running against an
appropriately scaled test catalog.



\subsubsection{Predecessors}

\subsubsection{Environment Needs}

\paragraph{Software}

\paragraph{Hardware}

\subsubsection{Input Specification}
QSERV has been set-up following procedure at ~LVV-T1017


\subsubsection{Output Specification}

\subsubsection{Test Procedure}
    \begin{longtable}[]{p{1.3cm}p{2cm}p{13cm}}
    %\toprule
    Step & \multicolumn{2}{@{}l}{Description, Input Data and Expected Result} \\ \toprule
    \endhead

            \multirow{3}{*}{ 1 } & Description &
            \begin{minipage}[t]{13cm}{\footnotesize
            Inspect and modify the CONCURRENCY and TARGET\_RATES dictionaries in the
runQueries.py script. ~Set CONCURRENCY and TARGET\_RATES for LV query
pool to 2020 value per table in LDM-552, section 2.2.1.~ Set CONCURRENCY
and TARGET\_RATES for all other query pools to values in next column
over from current year column (or to 2020 values +10\% if year is 2020)
per table in LDM-552, section 2.2.1.

            \vspace{\dp0}
            } \end{minipage} \\ \cline{2-3}
            & Test Data &
            \begin{minipage}[t]{13cm}{\footnotesize
                No data.
                \vspace{\dp0}
            } \end{minipage} \\ \cline{2-3}
            & Expected Result &
                \begin{minipage}[t]{13cm}{\footnotesize
                Script updated with appropriate values.

                \vspace{\dp0}
                } \end{minipage}
        \\ \midrule

            \multirow{3}{*}{ 2 } & Description &
            \begin{minipage}[t]{13cm}{\footnotesize
            Execute the runQueries.py script and let it run for 24 hrs.

            \vspace{\dp0}
            } \end{minipage} \\ \cline{2-3}
            & Test Data &
            \begin{minipage}[t]{13cm}{\footnotesize
                No data.
                \vspace{\dp0}
            } \end{minipage} \\ \cline{2-3}
            & Expected Result &
                \begin{minipage}[t]{13cm}{\footnotesize
                Script runs without error and produces output log.

                \vspace{\dp0}
                } \end{minipage}
        \\ \midrule

            \multirow{3}{*}{ 3 } & Description &
            \begin{minipage}[t]{13cm}{\footnotesize
            Examine log file output and compile average query execution times per
query type.

            \vspace{\dp0}
            } \end{minipage} \\ \cline{2-3}
            & Test Data &
            \begin{minipage}[t]{13cm}{\footnotesize
                No data.
                \vspace{\dp0}
            } \end{minipage} \\ \cline{2-3}
            & Expected Result &
                \begin{minipage}[t]{13cm}{\footnotesize
                Average query times per query type equal or less than corresponding
yearly target values in LDM-552, section 2.2.1.

                \vspace{\dp0}
                } \end{minipage}
        \\ \midrule
    \end{longtable}

\newpage
\section{Reusable Test Cases}

Test cases in this section are made up of commonly encountered steps that have been factored out into modular, reusable scripts.
These test cases are meant solely for the building of actual tests used for verification, to be inserted in test scripts via the “Call to Test” functionality in Jira/ATM.
They streamline the process of writing test scripts by providing pre-designed steps, while also ensuring homogeneity throughout the test suite.
These reusable modules are not themselves verifying requirements.
Also, these test cases shall not call other reusable test cases in their script.


  \textit{No library test cases found.}




\newpage
\section{Deprecated Test Cases}

This section includes all test cases that have been marked as deprecated.
These test cases will never be executed again, but have been in the past.
For this reason it is important to keep them in the baseline as a reference.

  \textit{No deprecated test cases found.}

\newpage
\appendix
