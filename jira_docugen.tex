% generated from JIRA project LVV
% using template at /var/jenkins_home/.local/lib/python3.7/site-packages/docsteady/templates/dm-spec.latex.jinja2.
% Collecting ATM data from folder: "/Data Management/Software Products/Supporting SW/Distrib Database|LDM-552"
% using docsteady version 1.2rc2
% Please do not edit -- update information in Jira instead

\section{Test Cases Summary}\label{test-cases-summary}

\begin{longtable}[]{p{3cm}p{13cm}}
\toprule
Test Id & Test Name\tabularnewline
\midrule
\endhead
    \hyperref[lvv-t1013]{LVV-T1013} &
    \href{https://jira.lsstcorp.org/secure/Tests.jspa\#/testCase/LVV-T1013}{QSERV-VER-00-00: Qserv design inspection} \tabularnewline
    \hyperref[lvv-t1014]{LVV-T1014} &
    \href{https://jira.lsstcorp.org/secure/Tests.jspa\#/testCase/LVV-T1014}{QSERV-VER-00-10: Qserv test inspection} \tabularnewline
    \hyperref[lvv-t1015]{LVV-T1015} &
    \href{https://jira.lsstcorp.org/secure/Tests.jspa\#/testCase/LVV-T1015}{QSERV-VER-00-05: Qserv code inspection} \tabularnewline
    \hyperref[lvv-t1016]{LVV-T1016} &
    \href{https://jira.lsstcorp.org/secure/Tests.jspa\#/testCase/LVV-T1016}{QSERV-PRF-20-00: Object Shared Scan Scaling} \tabularnewline
    \hyperref[lvv-t1017]{LVV-T1017} &
    \href{https://jira.lsstcorp.org/secure/Tests.jspa\#/testCase/LVV-T1017}{QSERV Preparation} \tabularnewline
    \hyperref[lvv-t1018]{LVV-T1018} &
    \href{https://jira.lsstcorp.org/secure/Tests.jspa\#/testCase/LVV-T1018}{Procedure QSERV-PRF-SCAN-SCALE-TEST} \tabularnewline
    \hyperref[lvv-t1019]{LVV-T1019} &
    \href{https://jira.lsstcorp.org/secure/Tests.jspa\#/testCase/LVV-T1019}{QSERV-PRF-20-05: Source Shared Scan Scaling} \tabularnewline
    \hyperref[lvv-t1020]{LVV-T1020} &
    \href{https://jira.lsstcorp.org/secure/Tests.jspa\#/testCase/LVV-T1020}{QSERV-PRF-20-10: Object Source Join Shared Scan Scaling} \tabularnewline
    \hyperref[lvv-t1021]{LVV-T1021} &
    \href{https://jira.lsstcorp.org/secure/Tests.jspa\#/testCase/LVV-T1021}{QSERV-PRF-20-15: Object ForcedSource Join Shared Scan Scaling} \tabularnewline
    \hyperref[lvv-t1022]{LVV-T1022} &
    \href{https://jira.lsstcorp.org/secure/Tests.jspa\#/testCase/LVV-T1022}{QSERV-PRF-10-00: Concurrent Query Performance} \tabularnewline
    \hyperref[lvv-t1085]{LVV-T1085} &
    \href{https://jira.lsstcorp.org/secure/Tests.jspa\#/testCase/LVV-T1085}{Short Queries Test} \tabularnewline
    \hyperref[lvv-t1086]{LVV-T1086} &
    \href{https://jira.lsstcorp.org/secure/Tests.jspa\#/testCase/LVV-T1086}{Full table scans, single query at a time} \tabularnewline
    \hyperref[lvv-t1087]{LVV-T1087} &
    \href{https://jira.lsstcorp.org/secure/Tests.jspa\#/testCase/LVV-T1087}{Full table joins, single query at a time} \tabularnewline
    \hyperref[lvv-t1088]{LVV-T1088} &
    \href{https://jira.lsstcorp.org/secure/Tests.jspa\#/testCase/LVV-T1088}{Concurrent scans and stress test} \tabularnewline
    \hyperref[lvv-t1089]{LVV-T1089} &
    \href{https://jira.lsstcorp.org/secure/Tests.jspa\#/testCase/LVV-T1089}{Load test 70 LV + 20 HV} \tabularnewline
    \hyperref[lvv-t1090]{LVV-T1090} &
    \href{https://jira.lsstcorp.org/secure/Tests.jspa\#/testCase/LVV-T1090}{Heavy load test 100 LV + 30 HV} \tabularnewline
    \hyperref[lvv-t1091]{LVV-T1091} &
    \href{https://jira.lsstcorp.org/secure/Tests.jspa\#/testCase/LVV-T1091}{SQL queries tests} \tabularnewline
\bottomrule
\end{longtable}

\newpage

\section{Test Cases}

\subsection{\href{https://jira.lsstcorp.org/secure/Tests.jspa\#/testCase/LVV-T1013}{LVV-T1013}
    - QSERV-VER-00-00: Qserv design inspection}\label{lvv-t1013}

\begin{longtable}[]{llllll}
\toprule
Version & Status & Priority & Verification Type & Owner
\\\midrule
1 & Draft & Normal &
Inspection & Fritz Mueller
\\\bottomrule
\end{longtable}

\subsubsection{Verification Elements}
    None.

\subsubsection{Test Items}
This test will check that the design of QSERV is adequate to meet the DM
subsystem requirements.\\
~\\



\subsubsection{Predecessors}

\subsubsection{Environment Needs}

\paragraph{Software}

\paragraph{Hardware}

\subsubsection{Input Specification}

\subsubsection{Output Specification}

\subsubsection{Test Procedure}
    \begin{longtable}[]{p{1.3cm}p{2cm}p{13cm}}
    %\toprule
    Step & \multicolumn{2}{@{}l}{Description, Input Data and Expected Result} \\ \toprule
    \endhead

            \multirow{3}{*}{ 1 } & Description &
            \begin{minipage}[t]{13cm}{\footnotesize
            Inspect that LDM-135 is up to date and that the design document is
adequate to fulfill the requirements documented in LDM-555.

            \vspace{\dp0}
            } \end{minipage} \\ \cline{2-3}
            & Test Data &
            \begin{minipage}[t]{13cm}{\footnotesize
                No data.
                \vspace{\dp0}
            } \end{minipage} \\ \cline{2-3}
            & Expected Result &
                \begin{minipage}[t]{13cm}{\footnotesize
                The design document in LDM-135 is adequate to fulfill the requirements
documented in LDM-555.

                \vspace{\dp0}
                } \end{minipage}
        \\ \midrule
    \end{longtable}

\subsection{\href{https://jira.lsstcorp.org/secure/Tests.jspa\#/testCase/LVV-T1014}{LVV-T1014}
    - QSERV-VER-00-10: Qserv test inspection}\label{lvv-t1014}

\begin{longtable}[]{llllll}
\toprule
Version & Status & Priority & Verification Type & Owner
\\\midrule
1 & Draft & Normal &
Inspection & Fritz Mueller
\\\bottomrule
\end{longtable}

\subsubsection{Verification Elements}
    None.

\subsubsection{Test Items}
This test will check:\\

\begin{itemize}
\tightlist
\item
  That all automated test suites associated with the product pass;
\item
  That there are no unexpected errors or warnings from the build, test
  or installation process.
\end{itemize}



\subsubsection{Predecessors}
LVV-T1015(QSERV-VER-00-05)


\subsubsection{Environment Needs}

\paragraph{Software}

\paragraph{Hardware}

\subsubsection{Input Specification}

\subsubsection{Output Specification}

\subsubsection{Test Procedure}
    \begin{longtable}[]{p{1.3cm}p{2cm}p{13cm}}
    %\toprule
    Step & \multicolumn{2}{@{}l}{Description, Input Data and Expected Result} \\ \toprule
    \endhead

            \multirow{3}{*}{ 1 } & Description &
            \begin{minipage}[t]{13cm}{\footnotesize
            Inspect the successful execution of the test suite.

            \vspace{\dp0}
            } \end{minipage} \\ \cline{2-3}
            & Test Data &
            \begin{minipage}[t]{13cm}{\footnotesize
                No data.
                \vspace{\dp0}
            } \end{minipage} \\ \cline{2-3}
            & Expected Result &
                \begin{minipage}[t]{13cm}{\footnotesize
                Unit tests are executed successfully and explanation shall be provided
for the skipped ones.

                \vspace{\dp0}
                } \end{minipage}
        \\ \midrule

            \multirow{3}{*}{ 2 } & Description &
            \begin{minipage}[t]{13cm}{\footnotesize
            Inspect the successful execution of the build process.

            \vspace{\dp0}
            } \end{minipage} \\ \cline{2-3}
            & Test Data &
            \begin{minipage}[t]{13cm}{\footnotesize
                No data.
                \vspace{\dp0}
            } \end{minipage} \\ \cline{2-3}
            & Expected Result &
                \begin{minipage}[t]{13cm}{\footnotesize
                No compiler, test, linter or other warnings associated with the software
build processing are found.

                \vspace{\dp0}
                } \end{minipage}
        \\ \midrule
    \end{longtable}

\subsection{\href{https://jira.lsstcorp.org/secure/Tests.jspa\#/testCase/LVV-T1015}{LVV-T1015}
    - QSERV-VER-00-05: Qserv code inspection}\label{lvv-t1015}

\begin{longtable}[]{llllll}
\toprule
Version & Status & Priority & Verification Type & Owner
\\\midrule
1 & Draft & Normal &
Inspection & Fritz Mueller
\\\bottomrule
\end{longtable}

\subsubsection{Verification Elements}
    None.

\subsubsection{Test Items}
This test will check:

\begin{itemize}
\tightlist
\item
  That the code delivered complies with relevant DM software quality
  standards;
\item
  That the code is accompanied by appropriate documentation;
\item
  That the code makes use of appropriate DM interfaces to the rest of
  the system where applicable;
\item
  That the code is appropriately tested.
\end{itemize}



\subsubsection{Predecessors}

\subsubsection{Environment Needs}

\paragraph{Software}

\paragraph{Hardware}

\subsubsection{Input Specification}

\subsubsection{Output Specification}

\subsubsection{Test Procedure}
    \begin{longtable}[]{p{1.3cm}p{2cm}p{13cm}}
    %\toprule
    Step & \multicolumn{2}{@{}l}{Description, Input Data and Expected Result} \\ \toprule
    \endhead

            \multirow{3}{*}{ 1 } & Description &
            \begin{minipage}[t]{13cm}{\footnotesize
            Check for the existence of a suite of unit test cases accompanying the
codebase

            \vspace{\dp0}
            } \end{minipage} \\ \cline{2-3}
            & Test Data &
            \begin{minipage}[t]{13cm}{\footnotesize
                No data.
                \vspace{\dp0}
            } \end{minipage} \\ \cline{2-3}
            & Expected Result &
                \begin{minipage}[t]{13cm}{\footnotesize
                Unit tests in the code base.

                \vspace{\dp0}
                } \end{minipage}
        \\ \midrule

            \multirow{3}{*}{ 2 } & Description &
            \begin{minipage}[t]{13cm}{\footnotesize
            Check the code to demonstrate that uses only standardized DM interfaces
for such things as logging and configuration (i.e. it does not print
directly to screen nor does it contain ad-hoc configuration parsers)

            \vspace{\dp0}
            } \end{minipage} \\ \cline{2-3}
            & Test Data &
            \begin{minipage}[t]{13cm}{\footnotesize
                No data.
                \vspace{\dp0}
            } \end{minipage} \\ \cline{2-3}
            & Expected Result &
                \begin{minipage}[t]{13cm}{\footnotesize
                DM standards are used.

                \vspace{\dp0}
                } \end{minipage}
        \\ \midrule

            \multirow{3}{*}{ 3 } & Description &
            \begin{minipage}[t]{13cm}{\footnotesize
            Check that the code is accompanied by a user manual describing
procedures for its installation and operation

            \vspace{\dp0}
            } \end{minipage} \\ \cline{2-3}
            & Test Data &
            \begin{minipage}[t]{13cm}{\footnotesize
                No data.
                \vspace{\dp0}
            } \end{minipage} \\ \cline{2-3}
            & Expected Result &
                \begin{minipage}[t]{13cm}{\footnotesize
                User manual is provided and up-to-date.

                \vspace{\dp0}
                } \end{minipage}
        \\ \midrule
    \end{longtable}

\subsection{\href{https://jira.lsstcorp.org/secure/Tests.jspa\#/testCase/LVV-T1016}{LVV-T1016}
    - QSERV-PRF-20-00: Object Shared Scan Scaling}\label{lvv-t1016}

\begin{longtable}[]{llllll}
\toprule
Version & Status & Priority & Verification Type & Owner
\\\midrule
1 & Approved & Normal &
Test & Fritz Mueller
\\\bottomrule
\end{longtable}

\subsubsection{Verification Elements}
\begin{itemize}
\item \href{https://jira.lsstcorp.org/browse/LVV-188}{LVV-188} - DMS-REQ-0357-V-01: Result latency for high-volume full-sky queries on
the Object table

\end{itemize}

\subsubsection{Test Items}
This test will show that average completion-time of full-scan queries of
the Object catalog table grow\\
sub-linearly with respect to the number of simultaneously active
full-scan queries, within the limits of\\
machine resource exhaustion.



\subsubsection{Predecessors}

\subsubsection{Environment Needs}

\paragraph{Software}

\paragraph{Hardware}

\subsubsection{Input Specification}
\begin{itemize}
\tightlist
\item
  A test catalog of appropriate size (see schedule detail in
  ~QSERV-PRF-10), prepared and ingested into the QSERV instance under
  test as detailed in LVV-T1017.
\item
  The concurrency load execution script, runQueries.py, maintained in
  the LSST QSERV github repository here:
  https://github.com/lsst/qserv/blob/master/admin/tools/docker/deployment/in2p3/runQueries.py
\end{itemize}


\subsubsection{Output Specification}
Log files as generated by the runQueries.py test script


\subsubsection{Test Procedure}
    \begin{longtable}[]{p{1.3cm}p{2cm}p{13cm}}
    %\toprule
    Step & \multicolumn{2}{@{}l}{Description, Input Data and Expected Result} \\ \toprule
    \endhead

                \multirow{3}{*}{\parbox{1.3cm}{ 1-1
                {\scriptsize from \hyperref[lvv-t1018]
                {LVV-T1018} } } }

                & {\small Description} &
                \begin{minipage}[t]{13cm}{\scriptsize
                Ensure that Qserv instance under test is up to date and that there is no
other concurrent user activity

                \vspace{\dp0}
                } \end{minipage} \\ \cdashline{2-3}
                & {\small Test Data} &
                \begin{minipage}[t]{13cm}{\scriptsize
                } \end{minipage} \\ \cdashline{2-3}
                & {\small Expected Result} &
                \\ \hdashline


                \multirow{3}{*}{\parbox{1.3cm}{ 1-2
                {\scriptsize from \hyperref[lvv-t1018]
                {LVV-T1018} } } }

                & {\small Description} &
                \begin{minipage}[t]{13cm}{\scriptsize
                Inspect and modify the \emph{CONCURRENCY} and \emph{TARGET\_RATES}
dictionaries in the run-Queries.py script. Set \emph{CONCURRENCY}
initially to 1 for the query pool of interest, and to 0 for all other
query pools. ~Set \emph{TARGET\_RATES} for the query pool of interest to
the yearly value per table in QSERV-PRF-10-00.

                \vspace{\dp0}
                } \end{minipage} \\ \cdashline{2-3}
                & {\small Test Data} &
                \begin{minipage}[t]{13cm}{\scriptsize
                } \end{minipage} \\ \cdashline{2-3}
                & {\small Expected Result} &
                \\ \hdashline


                \multirow{3}{*}{\parbox{1.3cm}{ 1-3
                {\scriptsize from \hyperref[lvv-t1018]
                {LVV-T1018} } } }

                & {\small Description} &
                \begin{minipage}[t]{13cm}{\scriptsize
                Execute the runQueries.py script and let it run for at least one, but
preferably several, query cycles.

                \vspace{\dp0}
                } \end{minipage} \\ \cdashline{2-3}
                & {\small Test Data} &
                \begin{minipage}[t]{13cm}{\scriptsize
                } \end{minipage} \\ \cdashline{2-3}
                & {\small Expected Result} &
                \\ \hdashline


                \multirow{3}{*}{\parbox{1.3cm}{ 1-4
                {\scriptsize from \hyperref[lvv-t1018]
                {LVV-T1018} } } }

                & {\small Description} &
                \begin{minipage}[t]{13cm}{\scriptsize
                Examine log file output and compile performance statistics to obtain a
growth curve point for the pool of interest for the test report.

                \vspace{\dp0}
                } \end{minipage} \\ \cdashline{2-3}
                & {\small Test Data} &
                \begin{minipage}[t]{13cm}{\scriptsize
                } \end{minipage} \\ \cdashline{2-3}
                & {\small Expected Result} &
                \\ \hdashline


                \multirow{3}{*}{\parbox{1.3cm}{ 1-5
                {\scriptsize from \hyperref[lvv-t1018]
                {LVV-T1018} } } }

                & {\small Description} &
                \begin{minipage}[t]{13cm}{\scriptsize
                Adjust the \emph{CONCURRENCY} value for the pool of interest and repeat
from the previous step to establish the growth trend and machine
resource exhaustion cutoff for the query pool of interest to an
acceptable degree of accuracy.

                \vspace{\dp0}
                } \end{minipage} \\ \cdashline{2-3}
                & {\small Test Data} &
                \begin{minipage}[t]{13cm}{\scriptsize
                } \end{minipage} \\ \cdashline{2-3}
                & {\small Expected Result} &
                \\ \hdashline


        \\ \midrule
    \end{longtable}

\subsection{\href{https://jira.lsstcorp.org/secure/Tests.jspa\#/testCase/LVV-T1017}{LVV-T1017}
    - QSERV Preparation}\label{lvv-t1017}

\begin{longtable}[]{llllll}
\toprule
Version & Status & Priority & Verification Type & Owner
\\\midrule
1 & Approved & Normal &
Test & Fritz Mueller
\\\bottomrule
\end{longtable}

\subsubsection{Verification Elements}
    None.

\subsubsection{Test Items}
Before running any of the performance test cases, Qserv must be
installed on an appropriate test cluster (e.g.\\
the test machine cluster at CC-IN2P3). ~To upgrade Qserv software on the
cluster in preparation for testing,\\
follow directions at
http://www.slac.stanford.edu/exp/lsst/qserv/2015\_10/HOW-TO/cluster-deployment.html.\\
~\\
The performance tests will also require an appropriately sized test
dataset to be synthesized and ingested,\\
per the yearly dataset sizing schedule described in LVV-T1022
(QSERV-PRF-10-00). ~Tools\\
for synthesis of ingest of test datasets may be found in the LSST github
repot at https://github.com/lsst-dm/db\_tests\_kpm20. ~Detailed use and
context information for the tools is described in
https://jira.lsstcorp.org/browse/DM-8405.\\
~\\
It has also been found that the Qserv shard servers must have
engine-independent statistics\\
loaded for the larger tables in the test dataset, and be properly\\
configured so that the MariaDB query planner can make use of those
statistics. ~More information on this\\
issue is available at
https://confluence.lsstcorp.org/pages/viewpage.action?pageId=58950786.



\subsubsection{Predecessors}

\subsubsection{Environment Needs}

\paragraph{Software}

\paragraph{Hardware}

\subsubsection{Input Specification}

\subsubsection{Output Specification}

\subsubsection{Test Procedure}
    \begin{longtable}[]{p{1.3cm}p{2cm}p{13cm}}
    %\toprule
    Step & \multicolumn{2}{@{}l}{Description, Input Data and Expected Result} \\ \toprule
    \endhead

            \multirow{3}{*}{ 1 } & Description &
            \begin{minipage}[t]{13cm}{\footnotesize
            Follow directions at
http://www.slac.stanford.edu/exp/lsst/qserv/2015\_10/HOW-TO/cluster-deployment.html

            \vspace{\dp0}
            } \end{minipage} \\ \cline{2-3}
            & Test Data &
            \begin{minipage}[t]{13cm}{\footnotesize
                No data.
                \vspace{\dp0}
            } \end{minipage} \\ \cline{2-3}
            & Expected Result &
                \begin{minipage}[t]{13cm}{\footnotesize
                Qserv installed.

                \vspace{\dp0}
                } \end{minipage}
        \\ \midrule
    \end{longtable}

\subsection{\href{https://jira.lsstcorp.org/secure/Tests.jspa\#/testCase/LVV-T1018}{LVV-T1018}
    - Procedure QSERV-PRF-SCAN-SCALE-TEST}\label{lvv-t1018}

\begin{longtable}[]{llllll}
\toprule
Version & Status & Priority & Verification Type & Owner
\\\midrule
1 & Approved & Normal &
Test & Fritz Mueller
\\\bottomrule
\end{longtable}

\subsubsection{Verification Elements}
    None.

\subsubsection{Test Items}
The objective of this procedure is to establish the growth trend for
average query execution time of\\
full-table-scan queries in the pool of interest, as a function of query
concurrency.~~The test shall\\
be considered passed if the growth rate is sub-linear (ideally, nearly
flat) within the limits of\\
machine resource exhaustion.



\subsubsection{Predecessors}

\subsubsection{Environment Needs}

\paragraph{Software}

\paragraph{Hardware}

\subsubsection{Input Specification}

\subsubsection{Output Specification}

\subsubsection{Test Procedure}
    \begin{longtable}[]{p{1.3cm}p{2cm}p{13cm}}
    %\toprule
    Step & \multicolumn{2}{@{}l}{Description, Input Data and Expected Result} \\ \toprule
    \endhead

            \multirow{3}{*}{ 1 } & Description &
            \begin{minipage}[t]{13cm}{\footnotesize
            Ensure that Qserv instance under test is up to date and that there is no
other concurrent user activity

            \vspace{\dp0}
            } \end{minipage} \\ \cline{2-3}
            & Test Data &
            \begin{minipage}[t]{13cm}{\footnotesize
                No data.
                \vspace{\dp0}
            } \end{minipage} \\ \cline{2-3}
            & Expected Result &
        \\ \midrule

            \multirow{3}{*}{ 2 } & Description &
            \begin{minipage}[t]{13cm}{\footnotesize
            Inspect and modify the \emph{CONCURRENCY} and \emph{TARGET\_RATES}
dictionaries in the run-Queries.py script. Set \emph{CONCURRENCY}
initially to 1 for the query pool of interest, and to 0 for all other
query pools. ~Set \emph{TARGET\_RATES} for the query pool of interest to
the yearly value per table in QSERV-PRF-10-00.

            \vspace{\dp0}
            } \end{minipage} \\ \cline{2-3}
            & Test Data &
            \begin{minipage}[t]{13cm}{\footnotesize
                No data.
                \vspace{\dp0}
            } \end{minipage} \\ \cline{2-3}
            & Expected Result &
        \\ \midrule

            \multirow{3}{*}{ 3 } & Description &
            \begin{minipage}[t]{13cm}{\footnotesize
            Execute the runQueries.py script and let it run for at least one, but
preferably several, query cycles.

            \vspace{\dp0}
            } \end{minipage} \\ \cline{2-3}
            & Test Data &
            \begin{minipage}[t]{13cm}{\footnotesize
                No data.
                \vspace{\dp0}
            } \end{minipage} \\ \cline{2-3}
            & Expected Result &
        \\ \midrule

            \multirow{3}{*}{ 4 } & Description &
            \begin{minipage}[t]{13cm}{\footnotesize
            Examine log file output and compile performance statistics to obtain a
growth curve point for the pool of interest for the test report.

            \vspace{\dp0}
            } \end{minipage} \\ \cline{2-3}
            & Test Data &
            \begin{minipage}[t]{13cm}{\footnotesize
                No data.
                \vspace{\dp0}
            } \end{minipage} \\ \cline{2-3}
            & Expected Result &
        \\ \midrule

            \multirow{3}{*}{ 5 } & Description &
            \begin{minipage}[t]{13cm}{\footnotesize
            Adjust the \emph{CONCURRENCY} value for the pool of interest and repeat
from the previous step to establish the growth trend and machine
resource exhaustion cutoff for the query pool of interest to an
acceptable degree of accuracy.

            \vspace{\dp0}
            } \end{minipage} \\ \cline{2-3}
            & Test Data &
            \begin{minipage}[t]{13cm}{\footnotesize
                No data.
                \vspace{\dp0}
            } \end{minipage} \\ \cline{2-3}
            & Expected Result &
        \\ \midrule
    \end{longtable}

\subsection{\href{https://jira.lsstcorp.org/secure/Tests.jspa\#/testCase/LVV-T1019}{LVV-T1019}
    - QSERV-PRF-20-05: Source Shared Scan Scaling}\label{lvv-t1019}

\begin{longtable}[]{llllll}
\toprule
Version & Status & Priority & Verification Type & Owner
\\\midrule
1 & Approved & Normal &
Test & Fritz Mueller
\\\bottomrule
\end{longtable}

\subsubsection{Verification Elements}
\begin{itemize}
\item \href{https://jira.lsstcorp.org/browse/LVV-188}{LVV-188} - DMS-REQ-0357-V-01: Result latency for high-volume full-sky queries on
the Object table

\end{itemize}

\subsubsection{Test Items}
This test will show that average completion-time of full-scan queries of
the Source catalog table grow sub-linearly with respect to the number of
simultaneously active full-scan queries, within the limits of machine
resource exhaustion.



\subsubsection{Predecessors}

\subsubsection{Environment Needs}

\paragraph{Software}

\paragraph{Hardware}

\subsubsection{Input Specification}
\begin{itemize}
\tightlist
\item
  A test catalog of appropriate size (see schedule detail in
  QSERV-PRF-10), prepared and ingested into the
  \textbackslash{}product\{\} instance under test as detailed in
  LVV-T1017.
\item
  The concurrency load execution script, runQueries.py, maintained in
  the LSST QSERV github repository here:
  https://github.com/lsst/qserv/blob/master/admin/tools/docker/deployment/in2p3/runQueries.py
\end{itemize}


\subsubsection{Output Specification}
Log files as generated by the runQueries.py test script.


\subsubsection{Test Procedure}
    \begin{longtable}[]{p{1.3cm}p{2cm}p{13cm}}
    %\toprule
    Step & \multicolumn{2}{@{}l}{Description, Input Data and Expected Result} \\ \toprule
    \endhead

                \multirow{3}{*}{\parbox{1.3cm}{ 1-1
                {\scriptsize from \hyperref[lvv-t1018]
                {LVV-T1018} } } }

                & {\small Description} &
                \begin{minipage}[t]{13cm}{\scriptsize
                Ensure that Qserv instance under test is up to date and that there is no
other concurrent user activity

                \vspace{\dp0}
                } \end{minipage} \\ \cdashline{2-3}
                & {\small Test Data} &
                \begin{minipage}[t]{13cm}{\scriptsize
                } \end{minipage} \\ \cdashline{2-3}
                & {\small Expected Result} &
                \\ \hdashline


                \multirow{3}{*}{\parbox{1.3cm}{ 1-2
                {\scriptsize from \hyperref[lvv-t1018]
                {LVV-T1018} } } }

                & {\small Description} &
                \begin{minipage}[t]{13cm}{\scriptsize
                Inspect and modify the \emph{CONCURRENCY} and \emph{TARGET\_RATES}
dictionaries in the run-Queries.py script. Set \emph{CONCURRENCY}
initially to 1 for the query pool of interest, and to 0 for all other
query pools. ~Set \emph{TARGET\_RATES} for the query pool of interest to
the yearly value per table in QSERV-PRF-10-00.

                \vspace{\dp0}
                } \end{minipage} \\ \cdashline{2-3}
                & {\small Test Data} &
                \begin{minipage}[t]{13cm}{\scriptsize
                } \end{minipage} \\ \cdashline{2-3}
                & {\small Expected Result} &
                \\ \hdashline


                \multirow{3}{*}{\parbox{1.3cm}{ 1-3
                {\scriptsize from \hyperref[lvv-t1018]
                {LVV-T1018} } } }

                & {\small Description} &
                \begin{minipage}[t]{13cm}{\scriptsize
                Execute the runQueries.py script and let it run for at least one, but
preferably several, query cycles.

                \vspace{\dp0}
                } \end{minipage} \\ \cdashline{2-3}
                & {\small Test Data} &
                \begin{minipage}[t]{13cm}{\scriptsize
                } \end{minipage} \\ \cdashline{2-3}
                & {\small Expected Result} &
                \\ \hdashline


                \multirow{3}{*}{\parbox{1.3cm}{ 1-4
                {\scriptsize from \hyperref[lvv-t1018]
                {LVV-T1018} } } }

                & {\small Description} &
                \begin{minipage}[t]{13cm}{\scriptsize
                Examine log file output and compile performance statistics to obtain a
growth curve point for the pool of interest for the test report.

                \vspace{\dp0}
                } \end{minipage} \\ \cdashline{2-3}
                & {\small Test Data} &
                \begin{minipage}[t]{13cm}{\scriptsize
                } \end{minipage} \\ \cdashline{2-3}
                & {\small Expected Result} &
                \\ \hdashline


                \multirow{3}{*}{\parbox{1.3cm}{ 1-5
                {\scriptsize from \hyperref[lvv-t1018]
                {LVV-T1018} } } }

                & {\small Description} &
                \begin{minipage}[t]{13cm}{\scriptsize
                Adjust the \emph{CONCURRENCY} value for the pool of interest and repeat
from the previous step to establish the growth trend and machine
resource exhaustion cutoff for the query pool of interest to an
acceptable degree of accuracy.

                \vspace{\dp0}
                } \end{minipage} \\ \cdashline{2-3}
                & {\small Test Data} &
                \begin{minipage}[t]{13cm}{\scriptsize
                } \end{minipage} \\ \cdashline{2-3}
                & {\small Expected Result} &
                \\ \hdashline


        \\ \midrule
    \end{longtable}

\subsection{\href{https://jira.lsstcorp.org/secure/Tests.jspa\#/testCase/LVV-T1020}{LVV-T1020}
    - QSERV-PRF-20-10: Object Source Join Shared Scan Scaling}\label{lvv-t1020}

\begin{longtable}[]{llllll}
\toprule
Version & Status & Priority & Verification Type & Owner
\\\midrule
1 & Approved & Normal &
Test & Fritz Mueller
\\\bottomrule
\end{longtable}

\subsubsection{Verification Elements}
\begin{itemize}
\item \href{https://jira.lsstcorp.org/browse/LVV-188}{LVV-188} - DMS-REQ-0357-V-01: Result latency for high-volume full-sky queries on
the Object table

\end{itemize}

\subsubsection{Test Items}
This test will show that average completion-time of full-scan queries
which join the Object and Source catalog tables grow sub-linearly with
respect to the number of simultaneously active full-scan queries, within
the limits of machine resource exhaustion.



\subsubsection{Predecessors}

\subsubsection{Environment Needs}

\paragraph{Software}

\paragraph{Hardware}

\subsubsection{Input Specification}
\begin{itemize}
\tightlist
\item
  A test catalog of appropriate size (see schedule detail in
  QSERV-PRF-10), prepared and ingested into the
  \textbackslash{}product\{\} instance under test as detailed in
  LVV-T1017.
\item
  The concurrency load execution script, runQueries.py, maintained in
  the LSST QSERV github repository here:
  https://github.com/lsst/qserv/blob/master/admin/tools/docker/deployment/in2p3/runQueries.py
\end{itemize}


\subsubsection{Output Specification}
Log files as generated by the runQueries.py test script


\subsubsection{Test Procedure}
    \begin{longtable}[]{p{1.3cm}p{2cm}p{13cm}}
    %\toprule
    Step & \multicolumn{2}{@{}l}{Description, Input Data and Expected Result} \\ \toprule
    \endhead

                \multirow{3}{*}{\parbox{1.3cm}{ 1-1
                {\scriptsize from \hyperref[lvv-t1018]
                {LVV-T1018} } } }

                & {\small Description} &
                \begin{minipage}[t]{13cm}{\scriptsize
                Ensure that Qserv instance under test is up to date and that there is no
other concurrent user activity

                \vspace{\dp0}
                } \end{minipage} \\ \cdashline{2-3}
                & {\small Test Data} &
                \begin{minipage}[t]{13cm}{\scriptsize
                } \end{minipage} \\ \cdashline{2-3}
                & {\small Expected Result} &
                \\ \hdashline


                \multirow{3}{*}{\parbox{1.3cm}{ 1-2
                {\scriptsize from \hyperref[lvv-t1018]
                {LVV-T1018} } } }

                & {\small Description} &
                \begin{minipage}[t]{13cm}{\scriptsize
                Inspect and modify the \emph{CONCURRENCY} and \emph{TARGET\_RATES}
dictionaries in the run-Queries.py script. Set \emph{CONCURRENCY}
initially to 1 for the query pool of interest, and to 0 for all other
query pools. ~Set \emph{TARGET\_RATES} for the query pool of interest to
the yearly value per table in QSERV-PRF-10-00.

                \vspace{\dp0}
                } \end{minipage} \\ \cdashline{2-3}
                & {\small Test Data} &
                \begin{minipage}[t]{13cm}{\scriptsize
                } \end{minipage} \\ \cdashline{2-3}
                & {\small Expected Result} &
                \\ \hdashline


                \multirow{3}{*}{\parbox{1.3cm}{ 1-3
                {\scriptsize from \hyperref[lvv-t1018]
                {LVV-T1018} } } }

                & {\small Description} &
                \begin{minipage}[t]{13cm}{\scriptsize
                Execute the runQueries.py script and let it run for at least one, but
preferably several, query cycles.

                \vspace{\dp0}
                } \end{minipage} \\ \cdashline{2-3}
                & {\small Test Data} &
                \begin{minipage}[t]{13cm}{\scriptsize
                } \end{minipage} \\ \cdashline{2-3}
                & {\small Expected Result} &
                \\ \hdashline


                \multirow{3}{*}{\parbox{1.3cm}{ 1-4
                {\scriptsize from \hyperref[lvv-t1018]
                {LVV-T1018} } } }

                & {\small Description} &
                \begin{minipage}[t]{13cm}{\scriptsize
                Examine log file output and compile performance statistics to obtain a
growth curve point for the pool of interest for the test report.

                \vspace{\dp0}
                } \end{minipage} \\ \cdashline{2-3}
                & {\small Test Data} &
                \begin{minipage}[t]{13cm}{\scriptsize
                } \end{minipage} \\ \cdashline{2-3}
                & {\small Expected Result} &
                \\ \hdashline


                \multirow{3}{*}{\parbox{1.3cm}{ 1-5
                {\scriptsize from \hyperref[lvv-t1018]
                {LVV-T1018} } } }

                & {\small Description} &
                \begin{minipage}[t]{13cm}{\scriptsize
                Adjust the \emph{CONCURRENCY} value for the pool of interest and repeat
from the previous step to establish the growth trend and machine
resource exhaustion cutoff for the query pool of interest to an
acceptable degree of accuracy.

                \vspace{\dp0}
                } \end{minipage} \\ \cdashline{2-3}
                & {\small Test Data} &
                \begin{minipage}[t]{13cm}{\scriptsize
                } \end{minipage} \\ \cdashline{2-3}
                & {\small Expected Result} &
                \\ \hdashline


        \\ \midrule
    \end{longtable}

\subsection{\href{https://jira.lsstcorp.org/secure/Tests.jspa\#/testCase/LVV-T1021}{LVV-T1021}
    - QSERV-PRF-20-15: Object ForcedSource Join Shared Scan Scaling}\label{lvv-t1021}

\begin{longtable}[]{llllll}
\toprule
Version & Status & Priority & Verification Type & Owner
\\\midrule
1 & Approved & Normal &
Test & Fritz Mueller
\\\bottomrule
\end{longtable}

\subsubsection{Verification Elements}
\begin{itemize}
\item \href{https://jira.lsstcorp.org/browse/LVV-188}{LVV-188} - DMS-REQ-0357-V-01: Result latency for high-volume full-sky queries on
the Object table

\end{itemize}

\subsubsection{Test Items}
This test will show that average completion-time of full-scan queries
which join the Object and ForcedSource catalog tables grow sub-linearly
with respect to the number of simultaneously active full-scan queries,
within the limits of machine resource exhaustion.



\subsubsection{Predecessors}

\subsubsection{Environment Needs}

\paragraph{Software}

\paragraph{Hardware}

\subsubsection{Input Specification}
\begin{itemize}
\tightlist
\item
  A test catalog of appropriate size (see schedule detail in
  QSERV-PRF-10), prepared and ingested into the
  \textbackslash{}product\{\} instance under test as detailed in
  LVV-T1017.
\item
  The concurrency load execution script, runQueries.py, maintained in
  the LSST QSERV github repository here:
  https://github.com/lsst/qserv/blob/master/admin/tools/docker/deployment/in2p3/runQueries.py
\end{itemize}


\subsubsection{Output Specification}
Log files as generated by the runQueries.py test script.


\subsubsection{Test Procedure}
    \begin{longtable}[]{p{1.3cm}p{2cm}p{13cm}}
    %\toprule
    Step & \multicolumn{2}{@{}l}{Description, Input Data and Expected Result} \\ \toprule
    \endhead

                \multirow{3}{*}{\parbox{1.3cm}{ 1-1
                {\scriptsize from \hyperref[lvv-t1018]
                {LVV-T1018} } } }

                & {\small Description} &
                \begin{minipage}[t]{13cm}{\scriptsize
                Ensure that Qserv instance under test is up to date and that there is no
other concurrent user activity

                \vspace{\dp0}
                } \end{minipage} \\ \cdashline{2-3}
                & {\small Test Data} &
                \begin{minipage}[t]{13cm}{\scriptsize
                } \end{minipage} \\ \cdashline{2-3}
                & {\small Expected Result} &
                \\ \hdashline


                \multirow{3}{*}{\parbox{1.3cm}{ 1-2
                {\scriptsize from \hyperref[lvv-t1018]
                {LVV-T1018} } } }

                & {\small Description} &
                \begin{minipage}[t]{13cm}{\scriptsize
                Inspect and modify the \emph{CONCURRENCY} and \emph{TARGET\_RATES}
dictionaries in the run-Queries.py script. Set \emph{CONCURRENCY}
initially to 1 for the query pool of interest, and to 0 for all other
query pools. ~Set \emph{TARGET\_RATES} for the query pool of interest to
the yearly value per table in QSERV-PRF-10-00.

                \vspace{\dp0}
                } \end{minipage} \\ \cdashline{2-3}
                & {\small Test Data} &
                \begin{minipage}[t]{13cm}{\scriptsize
                } \end{minipage} \\ \cdashline{2-3}
                & {\small Expected Result} &
                \\ \hdashline


                \multirow{3}{*}{\parbox{1.3cm}{ 1-3
                {\scriptsize from \hyperref[lvv-t1018]
                {LVV-T1018} } } }

                & {\small Description} &
                \begin{minipage}[t]{13cm}{\scriptsize
                Execute the runQueries.py script and let it run for at least one, but
preferably several, query cycles.

                \vspace{\dp0}
                } \end{minipage} \\ \cdashline{2-3}
                & {\small Test Data} &
                \begin{minipage}[t]{13cm}{\scriptsize
                } \end{minipage} \\ \cdashline{2-3}
                & {\small Expected Result} &
                \\ \hdashline


                \multirow{3}{*}{\parbox{1.3cm}{ 1-4
                {\scriptsize from \hyperref[lvv-t1018]
                {LVV-T1018} } } }

                & {\small Description} &
                \begin{minipage}[t]{13cm}{\scriptsize
                Examine log file output and compile performance statistics to obtain a
growth curve point for the pool of interest for the test report.

                \vspace{\dp0}
                } \end{minipage} \\ \cdashline{2-3}
                & {\small Test Data} &
                \begin{minipage}[t]{13cm}{\scriptsize
                } \end{minipage} \\ \cdashline{2-3}
                & {\small Expected Result} &
                \\ \hdashline


                \multirow{3}{*}{\parbox{1.3cm}{ 1-5
                {\scriptsize from \hyperref[lvv-t1018]
                {LVV-T1018} } } }

                & {\small Description} &
                \begin{minipage}[t]{13cm}{\scriptsize
                Adjust the \emph{CONCURRENCY} value for the pool of interest and repeat
from the previous step to establish the growth trend and machine
resource exhaustion cutoff for the query pool of interest to an
acceptable degree of accuracy.

                \vspace{\dp0}
                } \end{minipage} \\ \cdashline{2-3}
                & {\small Test Data} &
                \begin{minipage}[t]{13cm}{\scriptsize
                } \end{minipage} \\ \cdashline{2-3}
                & {\small Expected Result} &
                \\ \hdashline


        \\ \midrule
    \end{longtable}

\subsection{\href{https://jira.lsstcorp.org/secure/Tests.jspa\#/testCase/LVV-T1022}{LVV-T1022}
    - QSERV-PRF-10-00: Concurrent Query Performance}\label{lvv-t1022}

\begin{longtable}[]{llllll}
\toprule
Version & Status & Priority & Verification Type & Owner
\\\midrule
1 & Approved & Normal &
Test & Fritz Mueller
\\\bottomrule
\end{longtable}

\subsubsection{Verification Elements}
\begin{itemize}
\item \href{https://jira.lsstcorp.org/browse/LVV-188}{LVV-188} - DMS-REQ-0357-V-01: Result latency for high-volume full-sky queries on
the Object table

\item \href{https://jira.lsstcorp.org/browse/LVV-187}{LVV-187} - DMS-REQ-0356-V-01: Radius for low-volume query

\end{itemize}

\subsubsection{Test Items}
This test will check that QSERV is able to meet average query completion
time targets per query class\\
under a representative load of simultaneous high and low volume queries
while running against an appropriately\\
scaled test catalog.



\subsubsection{Predecessors}
LVV-T1016, LVV-T1019, LVV-T1020, LVV-T1021


\subsubsection{Environment Needs}

\paragraph{Software}

\paragraph{Hardware}

\subsubsection{Input Specification}
\begin{itemize}
\tightlist
\item
  A test catalog of appropriate size (see schedule detail in
  QSERV-PRF-10), prepared and ingested into the
  \textbackslash{}product\{\} instance under test as detailed in
  LVV-T1017.
\item
  The concurrency load execution script, runQueries.py, maintained in
  the LSST QSERV github repository here:
  https://github.com/lsst/qserv/blob/master/admin/tools/docker/deployment/in2p3/runQueries.py
\end{itemize}


\subsubsection{Output Specification}
Log files as generated by the runQueries.py test script


\subsubsection{Test Procedure}
    \begin{longtable}[]{p{1.3cm}p{2cm}p{13cm}}
    %\toprule
    Step & \multicolumn{2}{@{}l}{Description, Input Data and Expected Result} \\ \toprule
    \endhead

            \multirow{3}{*}{ 1 } & Description &
            \begin{minipage}[t]{13cm}{\footnotesize
            Inspect and possibly modify the \emph{CONCURRENCY} and
\emph{TARGET\_RATES} dictionaries in the \emph{runQueries.py} script to
adjust the concurrency mix and target execution times per query class.
~Query mixes and target times are to be adjusted per the following
schedule:\\
~\\
\textbf{LV Query Class~}(\#queries, \#time sec):~

\begin{itemize}
\tightlist
\item
  2015(50, 10) 2016(60, 10) 2017(70, 10) 2018(80, 10) 2019(90, 10)
  2020(100, 10)
\end{itemize}

\textbf{FTSObj Query Class~}(\#queries, \#time hours):~

\begin{itemize}
\tightlist
\item
  2015(3, 12) 2016(4, 1) 2017(8, 1) 2018(12, 1) 2019(16, 1) 2020(20, 1)
\end{itemize}

\textbf{FTSSrc Query Class~}(\#queries, \#time hours):~

\begin{itemize}
\tightlist
\item
  2015(1, 12) 2016(1, 12) 2017(2, 12) 2018(3, 12) 2019(4, 12) 2020(5,
  12)
\end{itemize}

\textbf{FTSFSrc Query Class~}(\#queries, \#time hours):~

\begin{itemize}
\tightlist
\item
  2015(0, 0) 2016(1, 12) 2017(2, 12) 2018(3, 12) 2019(4, 12) 2020(5, 12)
\end{itemize}

\textbf{joinObjSrc Query Class~}(\#queries, \#time hours):~

\begin{itemize}
\tightlist
\item
  2015(1, 12) 2016(2, 12) 2017(4, 12) 2018(6, 12) 2019(8, 12) 2020(10,
  12)
\end{itemize}

\textbf{joinObjFSrc Query Class~}(\#queries, \#time hours):~

\begin{itemize}
\tightlist
\item
  2015(0, 0) 2016(1, 12) 2017(2, 12) 2018(3, 12) 2019(4, 12) 2020(5, 12)
\end{itemize}

\textbf{nearN Query Class~}(\#queries, \#time hours):~

\begin{itemize}
\tightlist
\item
  2015(0, 0) 2016(1, 1) 2017(2, 1) 2018(3, 1) 2019(4, 1) 2020(5, 1)
\end{itemize}

            \vspace{\dp0}
            } \end{minipage} \\ \cline{2-3}
            & Test Data &
            \begin{minipage}[t]{13cm}{\footnotesize
                No data.
                \vspace{\dp0}
            } \end{minipage} \\ \cline{2-3}
            & Expected Result &
        \\ \midrule

            \multirow{3}{*}{ 2 } & Description &
            \begin{minipage}[t]{13cm}{\footnotesize
            Ensure that QSERV instance under test is up to date and that there is no
other concurrent user activity.\\
~\\

            \vspace{\dp0}
            } \end{minipage} \\ \cline{2-3}
            & Test Data &
            \begin{minipage}[t]{13cm}{\footnotesize
                No data.
                \vspace{\dp0}
            } \end{minipage} \\ \cline{2-3}
            & Expected Result &
        \\ \midrule

            \multirow{3}{*}{ 3 } & Description &
            \begin{minipage}[t]{13cm}{\footnotesize
            Execute the runQueries.py script and let it run for at least 24hrs.

            \vspace{\dp0}
            } \end{minipage} \\ \cline{2-3}
            & Test Data &
            \begin{minipage}[t]{13cm}{\footnotesize
                No data.
                \vspace{\dp0}
            } \end{minipage} \\ \cline{2-3}
            & Expected Result &
        \\ \midrule

            \multirow{3}{*}{ 4 } & Description &
            \begin{minipage}[t]{13cm}{\footnotesize
            Examine log file output and compile performance statistics for the test
report

            \vspace{\dp0}
            } \end{minipage} \\ \cline{2-3}
            & Test Data &
            \begin{minipage}[t]{13cm}{\footnotesize
                No data.
                \vspace{\dp0}
            } \end{minipage} \\ \cline{2-3}
            & Expected Result &
        \\ \midrule
    \end{longtable}

\subsection{\href{https://jira.lsstcorp.org/secure/Tests.jspa\#/testCase/LVV-T1085}{LVV-T1085}
    - Short Queries Test}\label{lvv-t1085}

\begin{longtable}[]{llllll}
\toprule
Version & Status & Priority & Verification Type & Owner
\\\midrule
1 & Draft & Normal &
Test & Fritz Mueller
\\\bottomrule
\end{longtable}

\subsubsection{Verification Elements}
\begin{itemize}
\item \href{https://jira.lsstcorp.org/browse/LVV-33}{LVV-33} - DMS-REQ-0075-V-01: Catalog Queries

\end{itemize}

\subsubsection{Test Items}
The objective of this test is to ensure that the short queries are
performing as expected



\subsubsection{Predecessors}

\subsubsection{Environment Needs}

\paragraph{Software}

\paragraph{Hardware}

\subsubsection{Input Specification}
QSERV has been set-up following procedure at ~LVV-T1017


\subsubsection{Output Specification}

\subsubsection{Test Procedure}
    \begin{longtable}[]{p{1.3cm}p{2cm}p{13cm}}
    %\toprule
    Step & \multicolumn{2}{@{}l}{Description, Input Data and Expected Result} \\ \toprule
    \endhead

            \multirow{3}{*}{ 1 } & Description &
            \begin{minipage}[t]{13cm}{\footnotesize
            Execute single object selection:\\
~\\
\textbf{SELECT} * \textbf{FROM} Object \textbf{WHERE} objectid =
\textless{}objId\textgreater{}\\
~\\
and record execution time.

            \vspace{\dp0}
            } \end{minipage} \\ \cline{2-3}
            & Test Data &
            \begin{minipage}[t]{13cm}{\footnotesize
                No data.
                \vspace{\dp0}
            } \end{minipage} \\ \cline{2-3}
            & Expected Result &
                \begin{minipage}[t]{13cm}{\footnotesize
                Query runs in less than 1 second.

                \vspace{\dp0}
                } \end{minipage}
        \\ \midrule

            \multirow{3}{*}{ 2 } & Description &
            \begin{minipage}[t]{13cm}{\footnotesize
            Execute spatial area selection from Object:\\
~\\
\textbf{SELECT} * \textbf{FROM} Object \textbf{WHERE}~\\

~qserv\_areaspec\_box(316.582327, −6.839078, 316.653938, −6.781822)

and record execution time.

            \vspace{\dp0}
            } \end{minipage} \\ \cline{2-3}
            & Test Data &
            \begin{minipage}[t]{13cm}{\footnotesize
                No data.
                \vspace{\dp0}
            } \end{minipage} \\ \cline{2-3}
            & Expected Result &
                \begin{minipage}[t]{13cm}{\footnotesize
                Query runs in less than 1 second.

                \vspace{\dp0}
                } \end{minipage}
        \\ \midrule
    \end{longtable}

\subsection{\href{https://jira.lsstcorp.org/secure/Tests.jspa\#/testCase/LVV-T1086}{LVV-T1086}
    - Full table scans, single query at a time}\label{lvv-t1086}

\begin{longtable}[]{llllll}
\toprule
Version & Status & Priority & Verification Type & Owner
\\\midrule
1 & Draft & Normal &
Test & Fritz Mueller
\\\bottomrule
\end{longtable}

\subsubsection{Verification Elements}
\begin{itemize}
\item \href{https://jira.lsstcorp.org/browse/LVV-33}{LVV-33} - DMS-REQ-0075-V-01: Catalog Queries

\end{itemize}

\subsubsection{Test Items}
Full table scans, single query at a time



\subsubsection{Predecessors}

\subsubsection{Environment Needs}

\paragraph{Software}

\paragraph{Hardware}

\subsubsection{Input Specification}
QSERV has been set-up following procedure at ~LVV-T1017


\subsubsection{Output Specification}

\subsubsection{Test Procedure}
    \begin{longtable}[]{p{1.3cm}p{2cm}p{13cm}}
    %\toprule
    Step & \multicolumn{2}{@{}l}{Description, Input Data and Expected Result} \\ \toprule
    \endhead

            \multirow{3}{*}{ 1 } & Description &
            \begin{minipage}[t]{13cm}{\footnotesize
            Execute query:\\
~\\
\textbf{SELECT} ra , decl , u\_psfFlux , g\_psfFlux , r\_psfFlux
\textbf{FROM} Object\\
\textbf{WHERE} y\_shapeIxx \textbf{BETWEEN} 20 \textbf{AND} 20.1\\
~\\
~\\
and record execution time and output size.

            \vspace{\dp0}
            } \end{minipage} \\ \cline{2-3}
            & Test Data &
            \begin{minipage}[t]{13cm}{\footnotesize
                No data.
                \vspace{\dp0}
            } \end{minipage} \\ \cline{2-3}
            & Expected Result &
                \begin{minipage}[t]{13cm}{\footnotesize
                Query expected to run in 20 minutes, output expected less then 100MB

                \vspace{\dp0}
                } \end{minipage}
        \\ \midrule

            \multirow{3}{*}{ 2 } & Description &
            \begin{minipage}[t]{13cm}{\footnotesize
            Execute query:\\
~\\
\textbf{SELECT} COUNT(*) \textbf{FROM} Source \textbf{WHERE} flux
\textbackslash{}\_sinc \textbf{BETWEEN} 1 \textbf{AND} 1.1\\
~\\
and record the execution time

            \vspace{\dp0}
            } \end{minipage} \\ \cline{2-3}
            & Test Data &
            \begin{minipage}[t]{13cm}{\footnotesize
                No data.
                \vspace{\dp0}
            } \end{minipage} \\ \cline{2-3}
            & Expected Result &
                \begin{minipage}[t]{13cm}{\footnotesize
                Query expected to run in 100 minutes

                \vspace{\dp0}
                } \end{minipage}
        \\ \midrule

            \multirow{3}{*}{ 3 } & Description &
            \begin{minipage}[t]{13cm}{\footnotesize
            Execute query:\\
~\\
\textbf{SELECT} COUNT(*) \textbf{FROM} ForcedSource \textbf{WHERE}
psfFlux \textbf{BETWEEN} 0.1 \textbf{AND} 0.2\\
~\\
and record the execution time

            \vspace{\dp0}
            } \end{minipage} \\ \cline{2-3}
            & Test Data &
            \begin{minipage}[t]{13cm}{\footnotesize
                No data.
                \vspace{\dp0}
            } \end{minipage} \\ \cline{2-3}
            & Expected Result &
                \begin{minipage}[t]{13cm}{\footnotesize
                Query expected to run in 50 minutes

                \vspace{\dp0}
                } \end{minipage}
        \\ \midrule
    \end{longtable}

\subsection{\href{https://jira.lsstcorp.org/secure/Tests.jspa\#/testCase/LVV-T1087}{LVV-T1087}
    - Full table joins, single query at a time}\label{lvv-t1087}

\begin{longtable}[]{llllll}
\toprule
Version & Status & Priority & Verification Type & Owner
\\\midrule
1 & Draft & Normal &
Test & Fritz Mueller
\\\bottomrule
\end{longtable}

\subsubsection{Verification Elements}
\begin{itemize}
\item \href{https://jira.lsstcorp.org/browse/LVV-33}{LVV-33} - DMS-REQ-0075-V-01: Catalog Queries

\end{itemize}

\subsubsection{Test Items}
TBC



\subsubsection{Predecessors}

\subsubsection{Environment Needs}

\paragraph{Software}

\paragraph{Hardware}

\subsubsection{Input Specification}
QSERV has been set-up following procedure at ~LVV-T1017


\subsubsection{Output Specification}

\subsubsection{Test Procedure}
    \begin{longtable}[]{p{1.3cm}p{2cm}p{13cm}}
    %\toprule
    Step & \multicolumn{2}{@{}l}{Description, Input Data and Expected Result} \\ \toprule
    \endhead

            \multirow{3}{*}{ 1 } & Description &
            \begin{minipage}[t]{13cm}{\footnotesize
            Execute query:\\
~\\
\textbf{SELECT} o.deepSourceId, s.objectId, s.id, o.ra, o.decl\\
\textbf{~ ~ FROM} Object o, Source s WHERE o.deepSourceId=s.objectId\\
\hspace*{0.333em} ~ \textbf{AND} s . flux\_sinc \textbf{BETWEEN} 0.3
\textbf{AND} 0.31\\
~\\
and record execution time.

            \vspace{\dp0}
            } \end{minipage} \\ \cline{2-3}
            & Test Data &
            \begin{minipage}[t]{13cm}{\footnotesize
                No data.
                \vspace{\dp0}
            } \end{minipage} \\ \cline{2-3}
            & Expected Result &
                \begin{minipage}[t]{13cm}{\footnotesize
                Expected to run as follows:

\begin{itemize}
\tightlist
\item
  2015: less than XXX
\item
  2016:~
\item
  2017:
\item
  2018:
\item
  2019:
\item
  2020:
\end{itemize}

                \vspace{\dp0}
                } \end{minipage}
        \\ \midrule

            \multirow{3}{*}{ 2 } & Description &
            \begin{minipage}[t]{13cm}{\footnotesize
            Execute query:\\
~\\
\textbf{SELECT} o.deepSourceId, f.psfFlux \textbf{FROM} Object o,
ForcedSource f\\
\textbf{~ ~ WHERE} o.deepSourceId=f.deepSourceId\\
\textbf{~ ~ AND} f . psfFlux \textbf{BETWEEN} 0.13 \textbf{AND} 0.14\\
~\\
and record execution time.

            \vspace{\dp0}
            } \end{minipage} \\ \cline{2-3}
            & Test Data &
            \begin{minipage}[t]{13cm}{\footnotesize
                No data.
                \vspace{\dp0}
            } \end{minipage} \\ \cline{2-3}
            & Expected Result &
                \begin{minipage}[t]{13cm}{\footnotesize
                to be specied

                \vspace{\dp0}
                } \end{minipage}
        \\ \midrule
    \end{longtable}

\subsection{\href{https://jira.lsstcorp.org/secure/Tests.jspa\#/testCase/LVV-T1088}{LVV-T1088}
    - Concurrent scans and stress test}\label{lvv-t1088}

\begin{longtable}[]{llllll}
\toprule
Version & Status & Priority & Verification Type & Owner
\\\midrule
1 & Draft & Normal &
Test & Fritz Mueller
\\\bottomrule
\end{longtable}

\subsubsection{Verification Elements}
\begin{itemize}
\item \href{https://jira.lsstcorp.org/browse/LVV-33}{LVV-33} - DMS-REQ-0075-V-01: Catalog Queries

\end{itemize}

\subsubsection{Test Items}
to be completed



\subsubsection{Predecessors}

\subsubsection{Environment Needs}

\paragraph{Software}

\paragraph{Hardware}

\subsubsection{Input Specification}
QSERV has been set-up following procedure at ~LVV-T1017


\subsubsection{Output Specification}

\subsubsection{Test Procedure}
    \begin{longtable}[]{p{1.3cm}p{2cm}p{13cm}}
    %\toprule
    Step & \multicolumn{2}{@{}l}{Description, Input Data and Expected Result} \\ \toprule
    \endhead

            \multirow{3}{*}{ 1 } & Description &
            \begin{minipage}[t]{13cm}{\footnotesize
            \emph{2 Object scans\\
}

            \vspace{\dp0}
            } \end{minipage} \\ \cline{2-3}
            & Test Data &
            \begin{minipage}[t]{13cm}{\footnotesize
                No data.
                \vspace{\dp0}
            } \end{minipage} \\ \cline{2-3}
            & Expected Result &
                \begin{minipage}[t]{13cm}{\footnotesize
                to be defined

                \vspace{\dp0}
                } \end{minipage}
        \\ \midrule

            \multirow{3}{*}{ 2 } & Description &
            \begin{minipage}[t]{13cm}{\footnotesize
            5 Object scans

            \vspace{\dp0}
            } \end{minipage} \\ \cline{2-3}
            & Test Data &
            \begin{minipage}[t]{13cm}{\footnotesize
                No data.
                \vspace{\dp0}
            } \end{minipage} \\ \cline{2-3}
            & Expected Result &
                \begin{minipage}[t]{13cm}{\footnotesize
                to be defined

                \vspace{\dp0}
                } \end{minipage}
        \\ \midrule

            \multirow{3}{*}{ 3 } & Description &
            \begin{minipage}[t]{13cm}{\footnotesize
            60 Object scans

            \vspace{\dp0}
            } \end{minipage} \\ \cline{2-3}
            & Test Data &
            \begin{minipage}[t]{13cm}{\footnotesize
                No data.
                \vspace{\dp0}
            } \end{minipage} \\ \cline{2-3}
            & Expected Result &
                \begin{minipage}[t]{13cm}{\footnotesize
                to be defined

                \vspace{\dp0}
                } \end{minipage}
        \\ \midrule
    \end{longtable}

\subsection{\href{https://jira.lsstcorp.org/secure/Tests.jspa\#/testCase/LVV-T1089}{LVV-T1089}
    - Load test 70 LV + 20 HV}\label{lvv-t1089}

\begin{longtable}[]{llllll}
\toprule
Version & Status & Priority & Verification Type & Owner
\\\midrule
1 & Draft & Normal &
Test & Fritz Mueller
\\\bottomrule
\end{longtable}

\subsubsection{Verification Elements}
\begin{itemize}
\item \href{https://jira.lsstcorp.org/browse/LVV-33}{LVV-33} - DMS-REQ-0075-V-01: Catalog Queries

\end{itemize}

\subsubsection{Test Items}
to be completed



\subsubsection{Predecessors}

\subsubsection{Environment Needs}

\paragraph{Software}

\paragraph{Hardware}

\subsubsection{Input Specification}
QSERV has been set-up following procedure at ~LVV-T1017


\subsubsection{Output Specification}

\subsubsection{Test Procedure}
    \begin{longtable}[]{p{1.3cm}p{2cm}p{13cm}}
    %\toprule
    Step & \multicolumn{2}{@{}l}{Description, Input Data and Expected Result} \\ \toprule
    \endhead

            \multirow{3}{*}{ 1 } & Description &
            \begin{minipage}[t]{13cm}{\footnotesize
            70 low volume and 20 high volume queries (8 scans for Object, 2 scans
for Source, 2 scans for ForcedSource, 4 Object-Source joins, 2
Object-ForcedSource joins and 2 NearNeigh- bor queries), all running
simultaneously with appropriate sleep in between queries to enforce the
mix we are aiming for\\
~\\

            \vspace{\dp0}
            } \end{minipage} \\ \cline{2-3}
            & Test Data &
            \begin{minipage}[t]{13cm}{\footnotesize
                No data.
                \vspace{\dp0}
            } \end{minipage} \\ \cline{2-3}
            & Expected Result &
                \begin{minipage}[t]{13cm}{\footnotesize
                to be defined

                \vspace{\dp0}
                } \end{minipage}
        \\ \midrule
    \end{longtable}

\subsection{\href{https://jira.lsstcorp.org/secure/Tests.jspa\#/testCase/LVV-T1090}{LVV-T1090}
    - Heavy load test 100 LV + 30 HV}\label{lvv-t1090}

\begin{longtable}[]{llllll}
\toprule
Version & Status & Priority & Verification Type & Owner
\\\midrule
1 & Draft & Normal &
Test & Fritz Mueller
\\\bottomrule
\end{longtable}

\subsubsection{Verification Elements}
\begin{itemize}
\item \href{https://jira.lsstcorp.org/browse/LVV-33}{LVV-33} - DMS-REQ-0075-V-01: Catalog Queries

\end{itemize}

\subsubsection{Test Items}
to be completed



\subsubsection{Predecessors}

\subsubsection{Environment Needs}

\paragraph{Software}

\paragraph{Hardware}

\subsubsection{Input Specification}
QSERV has been set-up following procedure at ~LVV-T1017


\subsubsection{Output Specification}

\subsubsection{Test Procedure}
    \begin{longtable}[]{p{1.3cm}p{2cm}p{13cm}}
    %\toprule
    Step & \multicolumn{2}{@{}l}{Description, Input Data and Expected Result} \\ \toprule
    \endhead

            \multirow{3}{*}{ 1 } & Description &
            \begin{minipage}[t]{13cm}{\footnotesize
            100 low volume and 30 high volume queries (12 scans for Object, 3 scans
for Source, 3 scans for ForcedSource, 6 Object-Source joins, 3
Object-ForcedSource join and 3 NearNeigh- bor queries), all running
simultaneously with appropriate sleep in between queries to enforce the
mix we are aiming for

            \vspace{\dp0}
            } \end{minipage} \\ \cline{2-3}
            & Test Data &
            \begin{minipage}[t]{13cm}{\footnotesize
                No data.
                \vspace{\dp0}
            } \end{minipage} \\ \cline{2-3}
            & Expected Result &
                \begin{minipage}[t]{13cm}{\footnotesize
                to be defined

                \vspace{\dp0}
                } \end{minipage}
        \\ \midrule
    \end{longtable}

\subsection{\href{https://jira.lsstcorp.org/secure/Tests.jspa\#/testCase/LVV-T1091}{LVV-T1091}
    - SQL queries tests}\label{lvv-t1091}

\begin{longtable}[]{llllll}
\toprule
Version & Status & Priority & Verification Type & Owner
\\\midrule
1 & Draft & Normal &
Test & Fritz Mueller
\\\bottomrule
\end{longtable}

\subsubsection{Verification Elements}
\begin{itemize}
\item \href{https://jira.lsstcorp.org/browse/LVV-33}{LVV-33} - DMS-REQ-0075-V-01: Catalog Queries

\end{itemize}

\subsubsection{Test Items}




\subsubsection{Predecessors}

\subsubsection{Environment Needs}

\paragraph{Software}

\paragraph{Hardware}

\subsubsection{Input Specification}

\subsubsection{Output Specification}

\subsubsection{Test Procedure}
    \begin{longtable}[]{p{1.3cm}p{2cm}p{13cm}}
    %\toprule
    Step & \multicolumn{2}{@{}l}{Description, Input Data and Expected Result} \\ \toprule
    \endhead

            \multirow{3}{*}{ 1 } & Description &
            \begin{minipage}[t]{13cm}{\footnotesize
            
            \vspace{\dp0}
            } \end{minipage} \\ \cline{2-3}
            & Test Data &
            \begin{minipage}[t]{13cm}{\footnotesize
                No data.
                \vspace{\dp0}
            } \end{minipage} \\ \cline{2-3}
            & Expected Result &
        \\ \midrule
    \end{longtable}

\appendix
