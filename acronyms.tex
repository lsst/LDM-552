\addtocounter{table}{-1}
\begin{longtable}{|l|p{0.8\textwidth}|}\hline
\textbf{Acronym} & \textbf{Description}  \\\hline

CC & Change Control \\\hline
CI & Continuous Integration \\\hline
DM & Data Management \\\hline
DMS & Data Management Subsystem \\\hline
DR & Data Release \\\hline
Data Access Center & Part of the LSST Data Management System, the US and Chilean DACs will provide authorized access to the released LSST data products, software such as the Science Platform, and computational resources for data analysis. The US DAC also includes a service for distributing bulk data on daily and annual (Data Release) timescales to partner institutions, collaborations, and LSST Education and Public Outreach (EPO).  \\\hline
Data Management & The LSST Subsystem responsible for the Data Management System (DMS), which will capture, store, catalog, and serve the LSST dataset to the scientific community and public. The DM team is responsible for the DMS architecture, applications, middleware, infrastructure, algorithms, and Observatory Network Design. DM is a distributed team working at LSST and partner institutions, with the DM Subsystem Manager located at LSST headquarters in Tucson. \\\hline
Data Release & The approximately annual reprocessing of all LSST data, and the installation of the resulting data products in the LSST Data Access Centers, which marks the start of the two-year proprietary period. \\\hline
Document & Any object (in any application supported by DocuShare or design archives such as PDMWorks or GIT) that supports project management or records milestones and deliverables of the LSST Project \\\hline
ForcedSource & DRP table resulting from forced photometry. \\\hline
HSC & Hyper Suprime-Cam \\\hline
IRSA & Infrared Science Archive \\\hline
LDM & LSST Data Management (Document Handle) \\\hline
LSST & Large Synoptic Survey Telescope \\\hline
NCSA & National Center for Supercomputing Applications \\\hline
Object & In LSST nomenclature this refers to an astronomical object, such as a star, galaxy, or other physical entity. E.g., comets, asteroids are also Objects but typically called a Moving Object or a Solar System Object (SSObject). One of the DRP data products is a table of Objects detected by LSST which can be static, or change brightness or position with time. \\\hline
PDAC & Prototype Data Access Center \\\hline
Qserv & LSST's distributed parallel database. This database server is used for collecting, storing, and serving LSST Data Release Catalogs and Project metadata, and is part of the Software Stack. \\\hline
RAM & Random Access Memory \\\hline
SDSS & Sloan Digital Sky Survey \\\hline
Scope & The work needed to be accomplished in order to deliver the product, service, or result with the specified features and functions \\\hline
Source & A single detection of an astrophysical object in an image, the characteristics for which are stored in the Source Catalog of the DRP database. The association of Sources that are non-moving lead to Objects; the association of moving Sources leads to Solar System Objects. (Note that in non-LSST usage "source" is often used for what LSST calls an Object.) \\\hline
Specification & One or more performance parameter(s) being established by a requirement that the delivered system or subsystem must meet \\\hline
Stripe 82 & A 2.5° wide equatorial band of sky covering roughly 300 square degrees that was observed repeatedly in 5 passbands during the course of the SDSS, In part for calibration purposes. \\\hline
TB & TeraByte \\\hline
Validation & A process of confirming that the delivered system will provide its desired functionality; overall, a validation process includes the evaluation, integration, and test activities carried out at the system level to ensure that the final developed system satisfies the intent and performance of that system in operations \\\hline
Verification & The process of evaluating the design, including hardware and software - to ensure the requirements have been met;  verification (of requirements) is performed by test, analysis, inspection, and/or demonstration \\\hline
monitoring & In DM QA, this refers to the process of collecting, storing, aggregating and visualizing metrics. \\\hline
stack & A record of all versions of a document uploaded to a particular DocuShare handle \\\hline
\end{longtable}
