\documentclass[DM,lsstdraft,STS,toc]{lsstdoc}
\usepackage{enumitem}
\input meta.tex

\begin{document}

\def\product{Qserv}

\setDocCompact{true}

\title[STS for \product]{\product~Software Test Specification}

\author{Fritz Mueller}
\setDocRef{\lsstDocType-\lsstDocNum}
\date{\vcsdate}

\setDocAbstract{
This document describes the detailed test specification for the \product.
}

\maketitle

% oldest first
\setDocChangeRecord{%
    \addtohist{1}{\vcsdate}{Initial release.}{FM}
}

\section{Introduction}
\label{sec:intro}

This document specifies the test procedure for the \product.

\subsection{Objectives}
\label{sec:objectives}

This document builds on the description of LSST Data Management's approach to
testing as described in \citeds{LDM-503} to describe the detailed tests that
will be performed on the \product as part of the verification of the DM system.

It identifies test designs, test cases and procedures for the tests, and the
pass/fail criteria for each test.

\subsection{Scope}
\label{sec:scope}

This document describes the test procedures for the following components of
the LSST system (as described in \citeds{LDM-148}):

\subsection{Applicable Documents}
\label{sec:docs}

\addtocounter{table}{-1}

\begin{tabular}[htb]{l l}
\citeds{LDM-294} & LSST DM Organization \& Management \\
\citeds{LDM-502} & The Measurement and Verification of DM Key Performance Metrics \\
\citeds{LDM-503} & LSST DM Test Plan \\
\citeds{LSE-61}  & LSST DM Subsystem Requirements \\
\citeds{LSE-163} & LSST Data Products Definition Document \\
\end{tabular}

\subsection{References\label{sect:references}}
\renewcommand{\refname}{}
\bibliography{lsst,refs,books,refs_ads}

\section{Approach}
\label{sec:approach}

Describe the approach to be utilized for the software testing specification. It should identify the major activities, methods and tools that are to be used to test the designated group of features.

\subsection{Tasks and criteria}
\label{sec:tasks}

Describe which are the items under tests, as well as criteria to be utilized. Activities should be described in sufficient detail to allow identification of the major testing tasks and estimation of the resources and time needed for the tests.

\subsection{Features to be tested}
\label{sec:feat2test}

Describe the GENERAL features to be tested.

\subsection{Features not to be tested}
\label{sec:featnot2test}

Describe all the features and significant combinations not to be tested and explain why. If it is not possible to test some features at their most appropriate level of testing but which will be tested at a later level, this information should be included here.

\subsection{Pass/fail criteria}
\label{sec:passfail}

GENERAL criteria to be used to determine whether or not tests are passed. If we agree its always what is in LDM-503 then this is not needed.

\subsection{Suspension criteria and resumption requirements}
\label{suspension}

Describe the criteria used to suspend all, or a part of, the testing activities on the test items associated withthe plan, as well as the activities to be repeated when testing is resumed.

\subsection{Naming convention}

All tests are named according to the pattern \textsc{prod-scope-xx-yy} where:

\begin{description}[font=\normalfont\scshape]

  \item[prod]{The product code, per \citeds{LDM-294}. Relevant entries for this document are:
    \begin{description}[font=\normalfont\scshape,topsep=-1.0ex]
      \item[qserv]{Qserv distributed database system}
    \end{description}
  }
  \item[scope]{The type of test being described:
    \begin{description}[font=\normalfont\scshape,topsep=-1.0ex]
      \item[fun]{concerning functional testing}
      \item[prf]{concerning performance testing}
      \item[int]{concerning integration testing}
      \item[mnt]{concerning maintenance testing}
      \item[acp]{concerning acceptance testing}
      \item[reg]{concerning regression testing}
      \item[ins]{concerning installation testing}
      \item[bck]{concerning backup and restore testing}
      \item[itf]{concerning interface testing}
    \end{description}
  }
  \item[xx]{Test design number (in increments of 10)}
  \item[yy]{Test case number (in increments of 5)}

\end{description}

\newpage
\section{Test Specification Design}
\subsection{\textsc{QSERV-VER-00}: Qserv Verification}
\label{qserv-ver-00}

\subsubsection{Objective}

This test verifies that \product{} as designed and built meets the overall requirements of the DM system.
Specifically, we verify that:

\begin{itemize}

  \item{Relevant requirements expressed in \citeds{LDM-555} are met by the design;}

  \item{The code as delivered is accompanied by a suite of unit tests;}

  \item{The code as delivered is accompanied by appropriate documentation;}

  \item{The code complies with all relevant DM coding standards\footnote{\url{https://developer.lsst.io/coding/intro.html}};}

  \item{The code makes use of standard DM interfaces to e.g. logging and configuration systems;}

  \item{The code is built and tested by the DM continuous integration system.}

\end{itemize}

\subsubsection{Approach refinements}

The general approach defined in \citeds{LDM-503} is used. Methods include:

\begin{itemize}
  \item{Document inspection;}
  \item{Code inspection;}
  \item{Review of CI system logs.}
\end{itemize}

\subsubsection{Test case identification}

\begin{longtable} {|p{0.4\textwidth}|p{0.6\textwidth}|}\hline
\textbf{Test Case}  & \textbf{Description} \\\hline
\hyperref[qserv-ver-00-00]{\textsc{QSERV-VER-00-00}} & Qserv design inspection \\\hline
\hyperref[qserv-ver-00-05]{\textsc{QSERV-VER-00-05}} & Qserv code inspection \\\hline
\hyperref[qserv-ver-00-10]{\textsc{QSERV-VER-00-10}} & Qserv test inspection \\\hline
\end{longtable}

\newpage
\subsection{\textsc{QSERV-PRF-10}: Qserv Concurrent Query Performance}
\label{qserv-prf-10}

\subsubsection{Objective}

This test design verifies that Qserv meets query concurrency performance requirements
per \citeds{LSE-61} and \citeds{LDM-555}.

\subsubsection{Approach refinements}

Query load on Qserv is anticipated to be a combination of "low volume" queries (queries that touch a 
small area of sky, or request a small number of objects) and "high volume" queries (queries that involve 
full-sky scans and may involve more complex spatial and temporal correlations).

Concurrency performance targets are expressed in terms of the allowable numbers of each of these query 
types that may be simultaneously active within the system while still meeting specified average query
completion times per type. 

The number of simultaneous low and high volume queries and the size of the dataset against which the
queries are issued are evolved along a glide path toward the eventual operation targets, provide a 
sequence of "data challenge" tests.  The following schedule shall be followed (FY20 LV and HV targets 
taken from \citeds{LSE-61}):

\begin{longtable}{|l|l|l|l|}\hline
  \textbf{Year}&\textbf{Dataset Size}&\textbf{\# LV Queries}&\textbf{\# HV Queries}\endhead\hline
  2015 & 10\% DR1  & 50  & 5  \\\hline
  2016 & 20\% DR1  & 60  & 10 \\\hline
  2017 & 30\% DR1  & 70  & 20 \\\hline
  2018 & 50\% DR1  & 80  & 30 \\\hline
  2019 & 75\% DR1  & 90  & 40 \\\hline
  2020 & 100\% DR1 & 100 & 50 \\\hline
\end{longtable}

\subsubsection{Test case identification}

\begin{longtable} {|p{0.4\textwidth}|p{0.6\textwidth}|}\hline
\textbf{Test Case}  & \textbf{Description} \\\hline
\hyperref[qserv-prf-10-00]{\textsc{QSERV-PRF-10-00}} & Qserv Concurrent Query Performance \\\hline
\end{longtable}

\newpage
\subsection{\textsc{QSERV-PRF-20}: Qserv Shared Scan Scaling}
\label{qserv-prf-20}

\subsubsection{Objective}

This test design verifies that the shared-scan feature of Qserv functions per design in \citeds{LDM-135},
allowing average completion-time of full-scan queries to grow sub-linearly with respect to the number of
simultaneously active full-scan queries, within the limits of machine resource exhaustion.

Proper behavior of Qserv in this regard is an important sub-goal of being able to feasibly meet high 
volume concurrency requirements as specified in \citeds{LSE-61}.  The outcome of these tests will also
help refine machine resource provisioning targets for operations.

\subsubsection{Approach refinements}

Full-scan queries will be simultaneously issued at increasing levels of concurrency while query 
completion times are monitored to establish the average query completion time growth trend.  The
tests will be carried out with full-scan queries of various types and levels of complexity in order
to verify that shared scan query classification and per-type scan scheduling function per design. 

\subsubsection{Test case identification}

\begin{longtable} {|p{0.4\textwidth}|p{0.6\textwidth}|}\hline
\textbf{Test Case}  & \textbf{Description} \\\hline
\hyperref[qserv-prf-20-00]{\textsc{QSERV-PRF-20-00}} & Object Shared Scan Scaling \\\hline
\hyperref[qserv-prf-20-05]{\textsc{QSERV-PRF-20-05}} & ObjectExtra Shared Scan Scaling \\\hline
\hyperref[qserv-prf-20-10]{\textsc{QSERV-PRF-20-10}} & Object Source Join Shared Scan Scaling \\\hline
\hyperref[qserv-prf-20-15]{\textsc{QSERV-PRF-20-15}} & Object FourcedSource Join Shared Scan Scaling \\\hline
\end{longtable}


\newpage
\section{Test Case Specification}

\subsection{Preparation}
\label{sec:prep}

Before running any of the performance test cases, Qserv must be installed on an appropriate test cluster (e.g.
the test machine cluster at CC-IN2P3).  To upgrade Qserv software on the cluster in preparation for testing,
follow directions at \url{http://www.slac.stanford.edu/exp/lsst/qserv/2015_10/HOW-TO/cluster-deployment.html}.

The performance tests will also require an appropriately sized test dataset to be synthesized and ingested,
per the yearly dataset sizing schedule described in \hyperref[qserv-prf-10]{\textsc{QSERV-PRF-10}}.  Tools
for synthesis of ingest of test datasets may be found in the LSST github repot at \url{https://github.com/lsst-dm/db_tests_kpm20}.  Detailed use and context information for the tools is described in \url{https://jira.lsstcorp.org/browse/DM-8405}.

It has also been found that the Qserv shard servers must have engine-independent statistics
loaded for the larger tables in the test dataset, and be properly
configured so that the MariaDB query planner can make use of those statistics.  More information on this
issue is avilable at \url{https://confluence.lsstcorp.org/pages/viewpage.action?pageId=58950786}.

\newpage
\subsection{\textsc{QSERV-VER-00-00}: Qserv design inspection}
\label{qserv-ver-00-00}

\subsubsection{Requirements}

\subsubsection{Test items}

This test will check that the design of \product{} is adequate to meet the DM subsystem requirements.

\subsubsection{Intercase dependencies}

None.

\subsubsection{Procedure}

By reference to \citeds{LDM-135}, the Qserv design document, and \citeds{LDM-555}, the Qserv requirements
document, demonstrate that elements exist in the design as stated to address each of individual requirements.
\newpage
\subsection{\textsc{QSERV-VER-00-05}: Qserv code inspection}
\label{qserv-ver-00-05}

\subsubsection{Requirements}

\subsubsection{Test items}

This test will check:

\begin{itemize}
  \item{That the code delivered complies with relevant DM software quality standards;}
  \item{That the code is accompanied by appropriate documentation;}
  \item{That the code makes use of appropriate DM interfaces to the rest of the system where applicable;}
  \item{That the code is appropriately tested.}
\end{itemize}

\subsubsection{Intercase dependencies}

None.

\subsubsection{Procedure}

\begin{itemize}

  \item{Check for the existence of a suite of unit test cases accompanying the codebase;}

  \item{Check the code to demonstrate that uses only standardized DM interfaces for such things as logging
  and configuration (i.e. it does not print directly to screen nor does it contain ad-hoc configuration
  parsers);}

  \item{Check that the code is accompanied by a user manual describing procedures for its installation and
  operation.}

\end{itemize}

\newpage
\subsection{\textsc{QSERV-VER-00-10}: Qserv test inspection}
\label{qserv-ver-00-10}

\subsubsection{Requirements}

\subsubsection{Test items}

This test will check:

\begin{itemize}
  \item{That all automated test suites associated with the product pass;}
  \item{That there are no unexpected errors or warnings from the build, test or installation process.}
\end{itemize}

\subsubsection{Intercase dependencies}

\hyperref[qserv-ver-00-05]{QSERV-VER-00-05}.

\subsubsection{Procedure}

Check the logs from the LSST CI system which was used to build and package the software under
test to ensure:

\begin{itemize}

  \item{Successful execution of the test suite, with no failures and no tests being skipped without
  explanatory documentation.}

  \item{That there were no compiler, test, linter or other warnings associated with the software
  build processing.}

\end{itemize}

\newpage
\subsection{\textsc{QSERV-PRF-10-00}: Concurrent Query Performance}
\label{qserv-prf-10-00}

\subsubsection{Requirements \label{sect:reqs}}

DMS-REQ-0356,DMS-REQ-0357.

\subsubsection{Test items}

This test will check that \product{} is able to meet average query completion time targets per query class
under a representative load of simultaneous high and low volume queries while running against an appropriately
scaled test catalog.

\subsubsection{Intercase dependencies}

\hyperref[qserv-prf-20-00]{\textsc{QSERV-PRF-20-00}},
\hyperref[qserv-prf-20-05]{\textsc{QSERV-PRF-20-05}},
\hyperref[qserv-prf-20-10]{\textsc{QSERV-PRF-20-10}},
\hyperref[qserv-prf-20-15]{\textsc{QSERV-PRF-20-15}}.

\subsubsection{Input specification}

\begin{itemize}

  \item{A test catalog of appropriate size (see schedule detail in \hyperref[qserv-prf-10]{\textsc{
  QSERV-PRF-10}}), prepared and ingested into the \product{} instance under test as detailed in
  section~\secref{sec:prep}.}

  \item{The concurrency load execution script, runQueries.py, maintained in the LSST \product{}
  github repository here: \url{https://github.com/lsst/qserv/blob/master/admin/tools/docker/deployment/in2p3/runQueries.py}}

\end{itemize}

\subsubsection{Output specification}

\begin{itemize}
  \item{Log files as generated by the runQueries.py test script.}
\end{itemize}

\subsubsection{Procedure}

\begin{enumerate}

  \item{Inspect and possibly modify the \texttt{CONCURRENCY} and \texttt{TARGET\_RATES} dictionaries in
  the runQueries.py script to adjust the concurrency mix and target execution times per query class.  Query
  mixes and target times are to be adjusted per the following schedule:

    \begin{tabular}{|l|c|c|c|c|c|c|c|}\hline
      \multicolumn{2}{|c|}{\textbf{Query Class}}
        &\textbf{2015}&\textbf{2016}&\textbf{2017}&\textbf{2018}&\textbf{2019}&\textbf{2020}\\\hline
      \multirow{2}{*}{\textbf{LV}}
        &\textbf{\# queries}  & 50 & 60 & 70 & 80 & 90 & 100 \\%\cline{2-8}
        &\textbf{time (sec)}  & 10 & 10 & 10 & 10 & 10 &  10 \\\hline
      \multirow{2}{*}{\textbf{FTSObj}}
        &\textbf{\# queries}  &  3 &  4 &  8 & 12 & 16 &  20 \\%\cline{2-8}
        &\textbf{time (hours)}& 12 &  1 &  1 &  1 &  1 &   1 \\\hline
      \multirow{2}{*}{\textbf{FTSSrc}}
        &\textbf{\# queries}  &  1 &  1 &  2 &  3 &  4 &   5 \\%\cline{2-8}
        &\textbf{time (hours)}& 12 & 12 & 12 & 12 & 12 &  12 \\\hline
      \multirow{2}{*}{\textbf{FTSFSrc}}
        &\textbf{\# queries}  &    &  1 &  2 &  3 &  4 &   5 \\%\cline{2-8}
        &\textbf{time (hours)}&    & 12 & 12 & 12 & 12 &  12 \\\hline
      \multirow{2}{*}{\textbf{joinObjSrc}}
        &\textbf{\# queries}  &  1 &  2 &  4 &  6 &  8 &  10 \\%\cline{2-8}
        &\textbf{time (hours)}& 12 & 12 & 12 & 12 & 12 &  12 \\\hline
      \multirow{2}{*}{\textbf{joinObjFSrc}}
        &\textbf{\# queries}  &    &  1 &  2 &  3 &  4 &   5 \\%\cline{2-8}
        &\textbf{time (hours)}&    & 12 & 12 & 12 & 12 &  12 \\\hline
      \multirow{2}{*}{\textbf{nearN}}
        &\textbf{\# queries}  &    &  1 &  2 &  3 &  4 &   5 \\%\cline{2-8}
        &\textbf{time (hours)}&    &  1 &  1 &  1 &  1 &   1 \\\hline
    \end{tabular}

  }

  \item{Ensure that \product{} instance under test is up to date and that there is no other concurrent
  user activity.}

  \item{Execute the runQueries.py script and let it run for at least 24hrs.}

  \item{Examine log file output and compile performance statistics for the test report.}

\end{enumerate}

\newpage
\input{cases/qserv-prf-20-00.tex}
\newpage
\input{cases/qserv-prf-20-05.tex}
\newpage
\input{cases/qserv-prf-20-10.tex}
\newpage
\subsection{\textsc{QSERV-PRF-20-15}: Object ForcedSource Join Shared Scan Scaling}
\label{qserv-prf-20-15}

\subsubsection{Requirements}

DMS-REQ-0357.

\subsubsection{Test items}

This test will show that average completion-time of full-scan queries which join the Object and ForcedSource
catalog tables grow sub-linearly with respect to the number of simultaneously active full-scan queries, within
the limits of machine resource exhaustion.

\subsubsection{Intercase dependencies}

None.

\subsubsection{Input specification}

\begin{itemize}

  \item{A test catalog of appropriate size (see schedule detail in \hyperref[qserv-prf-10]{\textsc{
  QSERV-PRF-10}}), prepared and ingested into the \product{} instance under test as detailed in
  section~\secref{sec:prep}.}

  \item{The concurrency load execution script, runQueries.py, maintained in the LSST \product{}
  github repository here: \url{https://github.com/lsst/qserv/blob/master/admin/tools/docker/deployment/in2p3/runQueries.py}}

\end{itemize}

\subsubsection{Output specification}

\begin{itemize}
  \item{Log files as generated by the runQueries.py test script.}
\end{itemize}

\subsubsection{Procedure}

\hyperref[qserv-prf-scan-scale-test]{\textsc{QSERV-PRF-SCAN-SCALE-TEST}}.
Query pool of interest is joinObjFSrc.


\end{document}
