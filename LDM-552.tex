\documentclass[DM,lsstdraft,STS,toc]{lsstdoc}
\usepackage{enumitem}
\usepackage{booktabs}
\usepackage{arydshln}

\input meta.tex

\setcounter{tocdepth}{2}

\def\product{Distributed Database}

\setDocCompact{true}

\title[STS for \product]{\product{} Software Test Specification}

\author{Fritz Mueller}
\setDocRef{LDM-552}
\date{\vcsdate}
\setDocUpstreamLocation{\url{https://github.com/lsst/LDM-552}}


\setDocAbstract{
This document describes the detailed test specification for the \product{}.
}

\setDocChangeRecord{%
    \addtohist{}{2019-06-14}{Autogenerate from Jira.}{G.~Comoretto}
    \addtohist{}{2017-07-02}{Initial draft.}{F.~Mueller}
}

\providecommand{\tightlist}{%
  \setlength{\itemsep}{0pt}\setlength{\parskip}{0pt}}

\begin{document}

\maketitle

\section{Introduction}
\label{sec:intro}

This document specifies the test procedure for \product{}. \product{} is a distributed
shared-nothing RDBMS which will host LSST catalogs.

\subsection{Objectives}
\label{sec:objectives}

This document builds on the description of LSST Data Management's approach to
testing as described in \citeds{LDM-503} to describe the detailed tests that
will be performed on the \product{} as part of the verification of the DM system.

It identifies test designs, test cases and procedures for the tests, and the
pass/fail criteria for each test.

\subsection{Scope}
\label{sec:scope}

This document describes the test procedures for the following components of
the LSST system (as described in \citeds{LDM-148}):

\begin{itemize}
  \item{Parallel Distributed Database (Qserv)}
\end{itemize}

\subsection{Applicable Documents}
\label{sec:docs}

\addtocounter{table}{-1}

\begin{tabular}[htb]{l l}
  \citeds{LDM-135} & LSST Qserv Database Design \\
  \citeds{LDM-294} & LSST DM Organization \& Management \\
  \citeds{LDM-502} & The Measurement and Verification of DM Key Performance Metrics \\
  \citeds{LDM-503} & LSST DM Test Plan \\
  \citeds{LDM-555} & LSST DM Database Requirements \\
\end{tabular}

\newpage
\section{Approach}
\label{sec:approach}

The approaches taken for the tests described here are:

\begin{itemize}

  \item{Ongoing inspection of design documents, code, and CI system logs to verify that \product{} design
  and implementation meet DM software quality standards in general, and requirements as expressed in
  \citeds{LDM-555} in particular;}

  \item{Ongoing deployment and continuous operation of \product{} in a Prototype Data Access Center
  (PDAC) in order to assess basic reliability, fitness for purpose, and integration with adjacent
  subsystems;}

  \item{Annual deployment of \product{} to test clusters, followed by synthesis and ingestion
  of test datasets and scripted performance/load/stress testing. The cluster size/capabilities and the
  scale of the synthetic test dataset are both evolved along a path toward anticipated LSST operational
  scale.}

\end{itemize}

\subsection{Tasks and criteria}
\label{sec:tasks}

\product{} is a containerized, distributed, Linux application, which is deployed on machine clusters.
At the scales to be tested, these clusters are comprised of one to several head ("czar") nodes and
additionally on the order of tens to hundreds of shard ("worker") nodes, interconnected locally via a
high-performance network. Head and shard nodes are provisioned each with on the order of 10s of gigabytes of
RAM, and each with on the order of 10s of terabytes of locally attached storage.

Ongoing deployment, continuous operation, and integration tests are carried out on machines within the
Prototype Data Access Center (PDAC), a dedicated machine cluster physically located at NCSA's National
Peta-scale Compute Facility, maintained by NCSA staff. Catalog datasets which are maintained within
the PDAC Qserv instance and which are used for this testing include, simultaneously:

\begin{itemize}
  \item{An LSST stack reprocessed version of the SDSS Stripe 82 catalog (currently from Summer 2013 \citedsp{Document-15097}) ({\textasciitilde{}}10 TB);}
  \item{IRSA AllWISE and NEOWISE catalogs ({\textasciitilde{}}50 TB);}
  \item{An LSST stack reprocessed version of the HSC catalog (scheduled; {\textasciitilde{}}50 TB).}
\end{itemize}

Tasks required for these tests include periodic update of the software deployed on the PDAC,
periodic ingest of additional test datasets, and inter-operation with adjacent subsystems.  Uptime
is monitored cumulatively throughout these activities to gain quantitative insight into system
stability and reliability.

Scaling, load, and stress testing are carried out on an additional machine cluster located
CC-IN2P3 in Lyon, maintained by CC-IN2P3 staff. Scaling tests are run annually, by issuing
a representative mix of concurrent queries against a synthetic catalog while monitoring
average query execution times per query type.  The scaling test dataset size and query
concurrency level are increased each year on a glide path toward the full scale of Data Release 1.

Tasks required for these tests include generation and ingest of each successive test dataset, and
execution of scripts which issue and monitor the suites of representative test queries.

\subsection{Features to be tested}
\label{sec:feat2test}

This version of the \product{} test specification addresses only basic product verification, basic
reliability, and performance/scale testing -- a bare minimum required to conduct ongoing development
and verify that \product{} remains on a realistic path towards meeting its most technically challenging
requirements: those related to successful operability at the scale that will be required by LSST.


\subsubsection{Performances}

In order to ensure that QSERV is able to meet the performance, specific test cases have been designed.
These test cases will be executed each year, in order to denmonstrate that query performances are as described in the following table.


    \begin{tabular}{|l|c|c|c|c|c|c|c|}\hline
      \multicolumn{2}{|c|}{\textbf{Query Class}}
        &\textbf{2015}&\textbf{2016}&\textbf{2017}&\textbf{2018}&\textbf{2019}&\textbf{2020}\\\hline
      \multirow{2}{*}{\textbf{LV}}
        &\textbf{\# queries}  & 50 & 60 & 70 & 80 & 90 & 100 \\%\cline{2-8}
        &\textbf{time (sec)}  & 10 & 10 & 10 & 10 & 10 &  10 \\\hline
      \multirow{2}{*}{\textbf{FTSObj}}
        &\textbf{\# queries}  &  3 &  4 &  8 & 12 & 16 &  20 \\%\cline{2-8}
        &\textbf{time (hours)}& 12 &  1 &  1 &  1 &  1 &   1 \\\hline
      \multirow{2}{*}{\textbf{FTSSrc}}
        &\textbf{\# queries}  &  1 &  1 &  2 &  3 &  4 &   5 \\%\cline{2-8}
        &\textbf{time (hours)}& 12 & 12 & 12 & 12 & 12 &  12 \\\hline
      \multirow{2}{*}{\textbf{FTSFSrc}}
        &\textbf{\# queries}  &    &  1 &  2 &  3 &  4 &   5 \\%\cline{2-8}
        &\textbf{time (hours)}&    & 12 & 12 & 12 & 12 &  12 \\\hline
      \multirow{2}{*}{\textbf{joinObjSrc}}
        &\textbf{\# queries}  &  1 &  2 &  4 &  6 &  8 &  10 \\%\cline{2-8}
        &\textbf{time (hours)}& 12 & 12 & 12 & 12 & 12 &  12 \\\hline
      \multirow{2}{*}{\textbf{joinObjFSrc}}
        &\textbf{\# queries}  &    &  1 &  2 &  3 &  4 &   5 \\%\cline{2-8}
        &\textbf{time (hours)}&    & 12 & 12 & 12 & 12 &  12 \\\hline
      \multirow{2}{*}{\textbf{nearN}}
        &\textbf{\# queries}  &    &  1 &  2 &  3 &  4 &   5 \\%\cline{2-8}
        &\textbf{time (hours)}&    &  1 &  1 &  1 &  1 &   1 \\\hline
    \end{tabular}


\subsection{Features not to be tested}
\label{sec:featnot2test}

Testing of the following are NOT YET COVERED in this specification:

\begin{itemize}
  \item{Fault-tolerance and disaster recovery;}
  \item{Schema evolution;}
  \item{Data ingest performance;}
  \item{Query reproducibility;}
  \item{Cross-match with external datasets.}
\end{itemize}

It is anticipated that test specifications and cases for all of the above will be developed
and added to future revisions of this document.

\subsection{Pass/fail criteria}
\label{sec:passfail}

The results of all tests will be assessed using the criteria described in \citeds{LDM-503} \S4.

\subsection{Suspension criteria and resumption requirements}
\label{suspension}

Refer to individual test cases where applicable.

\subsection{Naming convention}

With the introduction of the Jira Test Management, the following definitions have to be considered:

\begin{description}
  \item[LVV]{: Is the label for the ``LSST Verification and Validation'' project in Jira where all information regarding tests are managed.}
  \item[LVV-XXX]{: Are Verification Elements, where XXX is the Verification Element identifier.  Each Verification Element is derived from a requirement and has at least one Test Case associated. There can be multiple Verification Elements associated with a requirement.}
  \item[LVV-TYYY]{: Are Test Cases. Each Test Case is associated with a Verification Element, where YYY is the Test Case identifier. There can be multiple test cases associated with a Verification Element.}
\end{description}

All tests cases defined before the introduction of the Jira Test Management approach, are named according to the pattern \textsc{prod-scope-xx-yy} where:

\begin{description}[font=\normalfont\scshape]

  \item[prod]{The product code, per \citeds{LDM-294}. Relevant entries for this document are:
    \begin{description}[font=\normalfont\scshape,topsep=-1.0ex]
      \item[qserv]{Qserv distributed database system}
    \end{description}
  }

  \item[scope]{The type of test being described:
    \begin{description}[font=\normalfont\scshape,topsep=-1.0ex]
      \item[acp]{concerning acceptance testing}
      \item[bck]{concerning backup and restore testing}
      \item[fun]{concerning functional testing}
      \item[ins]{concerning installation testing}
      \item[int]{concerning integration testing}
      \item[itf]{concerning interface testing}
      \item[mnt]{concerning maintenance testing}
      \item[prf]{concerning performance testing}
      \item[reg]{concerning regression testing}
      \item[ver]{concerning verification testing}
    \end{description}
  }

  \item[xx]{Test design number (in increments of 10)}
  \item[yy]{Test case number (in increments of 5)}

\end{description}

% generated from JIRA project LVV
% using template at /var/jenkins_home/.local/lib/python3.7/site-packages/docsteady/templates/dm-spec.latex.jinja2.
% Collecting ATM data from folder: "/Data Management/Software Products/Supporting SW/Distrib Database|LDM-552"
% using docsteady version 1.2rc2
% Please do not edit -- update information in Jira instead

\section{Test Cases Summary}\label{test-cases-summary}

\begin{longtable}[]{p{3cm}p{13cm}}
\toprule
Test Id & Test Name\tabularnewline
\midrule
\endhead
    \hyperref[lvv-t1013]{LVV-T1013} &
    \href{https://jira.lsstcorp.org/secure/Tests.jspa\#/testCase/LVV-T1013}{QSERV-VER-00-00: Qserv design inspection} \tabularnewline
    \hyperref[lvv-t1014]{LVV-T1014} &
    \href{https://jira.lsstcorp.org/secure/Tests.jspa\#/testCase/LVV-T1014}{QSERV-VER-00-10: Qserv test inspection} \tabularnewline
    \hyperref[lvv-t1015]{LVV-T1015} &
    \href{https://jira.lsstcorp.org/secure/Tests.jspa\#/testCase/LVV-T1015}{QSERV-VER-00-05: Qserv code inspection} \tabularnewline
    \hyperref[lvv-t1016]{LVV-T1016} &
    \href{https://jira.lsstcorp.org/secure/Tests.jspa\#/testCase/LVV-T1016}{QSERV-PRF-20-00: Object Shared Scan Scaling} \tabularnewline
    \hyperref[lvv-t1017]{LVV-T1017} &
    \href{https://jira.lsstcorp.org/secure/Tests.jspa\#/testCase/LVV-T1017}{QSERV Preparation} \tabularnewline
    \hyperref[lvv-t1018]{LVV-T1018} &
    \href{https://jira.lsstcorp.org/secure/Tests.jspa\#/testCase/LVV-T1018}{Procedure QSERV-PRF-SCAN-SCALE-TEST} \tabularnewline
    \hyperref[lvv-t1019]{LVV-T1019} &
    \href{https://jira.lsstcorp.org/secure/Tests.jspa\#/testCase/LVV-T1019}{QSERV-PRF-20-05: Source Shared Scan Scaling} \tabularnewline
    \hyperref[lvv-t1020]{LVV-T1020} &
    \href{https://jira.lsstcorp.org/secure/Tests.jspa\#/testCase/LVV-T1020}{QSERV-PRF-20-10: Object Source Join Shared Scan Scaling} \tabularnewline
    \hyperref[lvv-t1021]{LVV-T1021} &
    \href{https://jira.lsstcorp.org/secure/Tests.jspa\#/testCase/LVV-T1021}{QSERV-PRF-20-15: Object ForcedSource Join Shared Scan Scaling} \tabularnewline
    \hyperref[lvv-t1022]{LVV-T1022} &
    \href{https://jira.lsstcorp.org/secure/Tests.jspa\#/testCase/LVV-T1022}{QSERV-PRF-10-00: Concurrent Query Performance} \tabularnewline
    \hyperref[lvv-t1085]{LVV-T1085} &
    \href{https://jira.lsstcorp.org/secure/Tests.jspa\#/testCase/LVV-T1085}{Short Queries Test} \tabularnewline
    \hyperref[lvv-t1086]{LVV-T1086} &
    \href{https://jira.lsstcorp.org/secure/Tests.jspa\#/testCase/LVV-T1086}{Full table scans, single query at a time} \tabularnewline
    \hyperref[lvv-t1087]{LVV-T1087} &
    \href{https://jira.lsstcorp.org/secure/Tests.jspa\#/testCase/LVV-T1087}{Full table joins, single query at a time} \tabularnewline
    \hyperref[lvv-t1088]{LVV-T1088} &
    \href{https://jira.lsstcorp.org/secure/Tests.jspa\#/testCase/LVV-T1088}{Concurrent scans and stress test} \tabularnewline
    \hyperref[lvv-t1089]{LVV-T1089} &
    \href{https://jira.lsstcorp.org/secure/Tests.jspa\#/testCase/LVV-T1089}{Load test 70 LV + 20 HV} \tabularnewline
    \hyperref[lvv-t1090]{LVV-T1090} &
    \href{https://jira.lsstcorp.org/secure/Tests.jspa\#/testCase/LVV-T1090}{Heavy load test 100 LV + 30 HV} \tabularnewline
\bottomrule
\end{longtable}

\newpage

\section{Test Cases}

\subsection{\href{https://jira.lsstcorp.org/secure/Tests.jspa\#/testCase/LVV-T1013}{LVV-T1013}
    - QSERV-VER-00-00: Qserv design inspection}\label{lvv-t1013}

\begin{longtable}[]{llllll}
\toprule
Version & Status & Priority & Verification Type & Owner
\\\midrule
1 & Draft & Normal &
Inspection & Fritz Mueller
\\\bottomrule
\end{longtable}

\subsubsection{Verification Elements}
    None.

\subsubsection{Test Items}
This test will check that the design of QSERV is adequate to meet the DM
subsystem requirements.\\
~\\



\subsubsection{Predecessors}

\subsubsection{Environment Needs}

\paragraph{Software}

\paragraph{Hardware}

\subsubsection{Input Specification}

\subsubsection{Output Specification}

\subsubsection{Test Procedure}
    \begin{longtable}[]{p{1.3cm}p{2cm}p{13cm}}
    %\toprule
    Step & \multicolumn{2}{@{}l}{Description, Input Data and Expected Result} \\ \toprule
    \endhead

            \multirow{3}{*}{ 1 } & Description &
            \begin{minipage}[t]{13cm}{\footnotesize
            Inspect that LDM-135 is up to date and that the design document is
adequate to fulfill the requirements documented in LDM-555.

            \vspace{\dp0}
            } \end{minipage} \\ \cline{2-3}
            & Test Data &
            \begin{minipage}[t]{13cm}{\footnotesize
                No data.
                \vspace{\dp0}
            } \end{minipage} \\ \cline{2-3}
            & Expected Result &
                \begin{minipage}[t]{13cm}{\footnotesize
                The design document in LDM-135 is adequate to fulfill the requirements
documented in LDM-555.

                \vspace{\dp0}
                } \end{minipage}
        \\ \midrule
    \end{longtable}

\subsection{\href{https://jira.lsstcorp.org/secure/Tests.jspa\#/testCase/LVV-T1014}{LVV-T1014}
    - QSERV-VER-00-10: Qserv test inspection}\label{lvv-t1014}

\begin{longtable}[]{llllll}
\toprule
Version & Status & Priority & Verification Type & Owner
\\\midrule
1 & Draft & Normal &
Inspection & Fritz Mueller
\\\bottomrule
\end{longtable}

\subsubsection{Verification Elements}
    None.

\subsubsection{Test Items}
This test will check:\\

\begin{itemize}
\tightlist
\item
  That all automated test suites associated with the product pass;
\item
  That there are no unexpected errors or warnings from the build, test
  or installation process.
\end{itemize}



\subsubsection{Predecessors}
LVV-T1015(QSERV-VER-00-05)


\subsubsection{Environment Needs}

\paragraph{Software}

\paragraph{Hardware}

\subsubsection{Input Specification}

\subsubsection{Output Specification}

\subsubsection{Test Procedure}
    \begin{longtable}[]{p{1.3cm}p{2cm}p{13cm}}
    %\toprule
    Step & \multicolumn{2}{@{}l}{Description, Input Data and Expected Result} \\ \toprule
    \endhead

            \multirow{3}{*}{ 1 } & Description &
            \begin{minipage}[t]{13cm}{\footnotesize
            Inspect the successful execution of the test suite.

            \vspace{\dp0}
            } \end{minipage} \\ \cline{2-3}
            & Test Data &
            \begin{minipage}[t]{13cm}{\footnotesize
                No data.
                \vspace{\dp0}
            } \end{minipage} \\ \cline{2-3}
            & Expected Result &
                \begin{minipage}[t]{13cm}{\footnotesize
                Unit tests are executed successfully and explanation shall be provided
for the skipped ones.

                \vspace{\dp0}
                } \end{minipage}
        \\ \midrule

            \multirow{3}{*}{ 2 } & Description &
            \begin{minipage}[t]{13cm}{\footnotesize
            Inspect the successful execution of the build process.

            \vspace{\dp0}
            } \end{minipage} \\ \cline{2-3}
            & Test Data &
            \begin{minipage}[t]{13cm}{\footnotesize
                No data.
                \vspace{\dp0}
            } \end{minipage} \\ \cline{2-3}
            & Expected Result &
                \begin{minipage}[t]{13cm}{\footnotesize
                No compiler, test, linter or other warnings associated with the software
build processing are found.

                \vspace{\dp0}
                } \end{minipage}
        \\ \midrule
    \end{longtable}

\subsection{\href{https://jira.lsstcorp.org/secure/Tests.jspa\#/testCase/LVV-T1015}{LVV-T1015}
    - QSERV-VER-00-05: Qserv code inspection}\label{lvv-t1015}

\begin{longtable}[]{llllll}
\toprule
Version & Status & Priority & Verification Type & Owner
\\\midrule
1 & Approved & Normal &
Inspection & Fritz Mueller
\\\bottomrule
\end{longtable}

\subsubsection{Verification Elements}
    None.

\subsubsection{Test Items}
This test will check:

\begin{itemize}
\tightlist
\item
  That the code delivered complies with relevant DM software quality
  standards;
\item
  That the code is accompanied by appropriate documentation;
\item
  That the code makes use of appropriate DM interfaces to the rest of
  the system where applicable;
\item
  That the code is appropriately tested.
\end{itemize}



\subsubsection{Predecessors}

\subsubsection{Environment Needs}

\paragraph{Software}

\paragraph{Hardware}

\subsubsection{Input Specification}

\subsubsection{Output Specification}

\subsubsection{Test Procedure}
    \begin{longtable}[]{p{1.3cm}p{2cm}p{13cm}}
    %\toprule
    Step & \multicolumn{2}{@{}l}{Description, Input Data and Expected Result} \\ \toprule
    \endhead

            \multirow{3}{*}{ 1 } & Description &
            \begin{minipage}[t]{13cm}{\footnotesize
            Check for the existence of a suite of unit test cases accompanying the
codebase

            \vspace{\dp0}
            } \end{minipage} \\ \cline{2-3}
            & Test Data &
            \begin{minipage}[t]{13cm}{\footnotesize
                No data.
                \vspace{\dp0}
            } \end{minipage} \\ \cline{2-3}
            & Expected Result &
                \begin{minipage}[t]{13cm}{\footnotesize
                Unit tests in the code base.

                \vspace{\dp0}
                } \end{minipage}
        \\ \midrule

            \multirow{3}{*}{ 2 } & Description &
            \begin{minipage}[t]{13cm}{\footnotesize
            Check the code to demonstrate that uses only standardized DM interfaces
for such things as logging and configuration (i.e. it does not print
directly to screen nor does it contain ad-hoc configuration parsers)

            \vspace{\dp0}
            } \end{minipage} \\ \cline{2-3}
            & Test Data &
            \begin{minipage}[t]{13cm}{\footnotesize
                No data.
                \vspace{\dp0}
            } \end{minipage} \\ \cline{2-3}
            & Expected Result &
                \begin{minipage}[t]{13cm}{\footnotesize
                DM standards are used.

                \vspace{\dp0}
                } \end{minipage}
        \\ \midrule

            \multirow{3}{*}{ 3 } & Description &
            \begin{minipage}[t]{13cm}{\footnotesize
            Check that the code is accompanied by a user manual describing
procedures for its installation and operation

            \vspace{\dp0}
            } \end{minipage} \\ \cline{2-3}
            & Test Data &
            \begin{minipage}[t]{13cm}{\footnotesize
                No data.
                \vspace{\dp0}
            } \end{minipage} \\ \cline{2-3}
            & Expected Result &
                \begin{minipage}[t]{13cm}{\footnotesize
                User manual is provided and up-to-date.

                \vspace{\dp0}
                } \end{minipage}
        \\ \midrule
    \end{longtable}

\subsection{\href{https://jira.lsstcorp.org/secure/Tests.jspa\#/testCase/LVV-T1016}{LVV-T1016}
    - QSERV-PRF-20-00: Object Shared Scan Scaling}\label{lvv-t1016}

\begin{longtable}[]{llllll}
\toprule
Version & Status & Priority & Verification Type & Owner
\\\midrule
1 & Approved & Normal &
Test & Fritz Mueller
\\\bottomrule
\end{longtable}

\subsubsection{Verification Elements}
\begin{itemize}
\item \href{https://jira.lsstcorp.org/browse/LVV-188}{LVV-188} - DMS-REQ-0357-V-01: Result latency for high-volume full-sky queries on
the Object table

\end{itemize}

\subsubsection{Test Items}
This test will show that average completion-time of full-scan queries of
the Object catalog table grow\\
sub-linearly with respect to the number of simultaneously active
full-scan queries, within the limits of\\
machine resource exhaustion.



\subsubsection{Predecessors}

\subsubsection{Environment Needs}

\paragraph{Software}

\paragraph{Hardware}

\subsubsection{Input Specification}
\begin{itemize}
\tightlist
\item
  A test catalog of appropriate size (see schedule detail in
  ~QSERV-PRF-10), prepared and ingested into the QSERV instance under
  test as detailed in LVV-T1017.
\item
  The concurrency load execution script, runQueries.py, maintained in
  the LSST QSERV github repository here:
  https://github.com/lsst/qserv/blob/master/admin/tools/docker/deployment/in2p3/runQueries.py
\end{itemize}


\subsubsection{Output Specification}
Log files as generated by the runQueries.py test script


\subsubsection{Test Procedure}
    \begin{longtable}[]{p{1.3cm}p{2cm}p{13cm}}
    %\toprule
    Step & \multicolumn{2}{@{}l}{Description, Input Data and Expected Result} \\ \toprule
    \endhead

                \multirow{3}{*}{\parbox{1.3cm}{ 1-1
                {\scriptsize from \hyperref[lvv-t1018]
                {LVV-T1018} } } }

                & {\small Description} &
                \begin{minipage}[t]{13cm}{\scriptsize
                Ensure that Qserv instance under test is up to date and that there is no
other concurrent user activity

                \vspace{\dp0}
                } \end{minipage} \\ \cdashline{2-3}
                & {\small Test Data} &
                \begin{minipage}[t]{13cm}{\scriptsize
                } \end{minipage} \\ \cdashline{2-3}
                & {\small Expected Result} &
                \\ \hdashline


                \multirow{3}{*}{\parbox{1.3cm}{ 1-2
                {\scriptsize from \hyperref[lvv-t1018]
                {LVV-T1018} } } }

                & {\small Description} &
                \begin{minipage}[t]{13cm}{\scriptsize
                Inspect and modify the \emph{CONCURRENCY} and \emph{TARGET\_RATES}
dictionaries in the run-Queries.py script. Set \emph{CONCURRENCY}
initially to 1 for the query pool of interest, and to 0 for all other
query pools. ~Set \emph{TARGET\_RATES} for the query pool of interest to
the yearly value per table in QSERV-PRF-10-00.

                \vspace{\dp0}
                } \end{minipage} \\ \cdashline{2-3}
                & {\small Test Data} &
                \begin{minipage}[t]{13cm}{\scriptsize
                } \end{minipage} \\ \cdashline{2-3}
                & {\small Expected Result} &
                \\ \hdashline


                \multirow{3}{*}{\parbox{1.3cm}{ 1-3
                {\scriptsize from \hyperref[lvv-t1018]
                {LVV-T1018} } } }

                & {\small Description} &
                \begin{minipage}[t]{13cm}{\scriptsize
                Execute the runQueries.py script and let it run for at least one, but
preferably several, query cycles.

                \vspace{\dp0}
                } \end{minipage} \\ \cdashline{2-3}
                & {\small Test Data} &
                \begin{minipage}[t]{13cm}{\scriptsize
                } \end{minipage} \\ \cdashline{2-3}
                & {\small Expected Result} &
                \\ \hdashline


                \multirow{3}{*}{\parbox{1.3cm}{ 1-4
                {\scriptsize from \hyperref[lvv-t1018]
                {LVV-T1018} } } }

                & {\small Description} &
                \begin{minipage}[t]{13cm}{\scriptsize
                Examine log file output and compile performance statistics to obtain a
growth curve point for the pool of interest for the test report.

                \vspace{\dp0}
                } \end{minipage} \\ \cdashline{2-3}
                & {\small Test Data} &
                \begin{minipage}[t]{13cm}{\scriptsize
                } \end{minipage} \\ \cdashline{2-3}
                & {\small Expected Result} &
                \\ \hdashline


                \multirow{3}{*}{\parbox{1.3cm}{ 1-5
                {\scriptsize from \hyperref[lvv-t1018]
                {LVV-T1018} } } }

                & {\small Description} &
                \begin{minipage}[t]{13cm}{\scriptsize
                Adjust the \emph{CONCURRENCY} value for the pool of interest and repeat
from the previous step to establish the growth trend and machine
resource exhaustion cutoff for the query pool of interest to an
acceptable degree of accuracy.

                \vspace{\dp0}
                } \end{minipage} \\ \cdashline{2-3}
                & {\small Test Data} &
                \begin{minipage}[t]{13cm}{\scriptsize
                } \end{minipage} \\ \cdashline{2-3}
                & {\small Expected Result} &
                \\ \hdashline


        \\ \midrule
    \end{longtable}

\subsection{\href{https://jira.lsstcorp.org/secure/Tests.jspa\#/testCase/LVV-T1017}{LVV-T1017}
    - QSERV Preparation}\label{lvv-t1017}

\begin{longtable}[]{llllll}
\toprule
Version & Status & Priority & Verification Type & Owner
\\\midrule
1 & Approved & Normal &
Test & Fritz Mueller
\\\bottomrule
\end{longtable}

\subsubsection{Verification Elements}
    None.

\subsubsection{Test Items}
Before running any of the performance test cases, Qserv must be
installed on an appropriate test cluster (e.g.\\
the test machine cluster at CC-IN2P3). ~To upgrade Qserv software on the
cluster in preparation for testing,\\
follow directions at
http://www.slac.stanford.edu/exp/lsst/qserv/2015\_10/HOW-TO/cluster-deployment.html.\\
~\\
The performance tests will also require an appropriately sized test
dataset to be synthesized and ingested,\\
per the yearly dataset sizing schedule described in LVV-T1022
(QSERV-PRF-10-00). ~Tools\\
for synthesis of ingest of test datasets may be found in the LSST github
repot at https://github.com/lsst-dm/db\_tests\_kpm20. ~Detailed use and
context information for the tools is described in
https://jira.lsstcorp.org/browse/DM-8405.\\
~\\
It has also been found that the Qserv shard servers must have
engine-independent statistics\\
loaded for the larger tables in the test dataset, and be properly\\
configured so that the MariaDB query planner can make use of those
statistics. ~More information on this\\
issue is available at
https://confluence.lsstcorp.org/pages/viewpage.action?pageId=58950786.



\subsubsection{Predecessors}

\subsubsection{Environment Needs}

\paragraph{Software}

\paragraph{Hardware}

\subsubsection{Input Specification}

\subsubsection{Output Specification}

\subsubsection{Test Procedure}
    \begin{longtable}[]{p{1.3cm}p{2cm}p{13cm}}
    %\toprule
    Step & \multicolumn{2}{@{}l}{Description, Input Data and Expected Result} \\ \toprule
    \endhead

            \multirow{3}{*}{ 1 } & Description &
            \begin{minipage}[t]{13cm}{\footnotesize
            Follow directions at
http://www.slac.stanford.edu/exp/lsst/qserv/2015\_10/HOW-TO/cluster-deployment.html

            \vspace{\dp0}
            } \end{minipage} \\ \cline{2-3}
            & Test Data &
            \begin{minipage}[t]{13cm}{\footnotesize
                No data.
                \vspace{\dp0}
            } \end{minipage} \\ \cline{2-3}
            & Expected Result &
                \begin{minipage}[t]{13cm}{\footnotesize
                Qserv installed.

                \vspace{\dp0}
                } \end{minipage}
        \\ \midrule
    \end{longtable}

\subsection{\href{https://jira.lsstcorp.org/secure/Tests.jspa\#/testCase/LVV-T1018}{LVV-T1018}
    - Procedure QSERV-PRF-SCAN-SCALE-TEST}\label{lvv-t1018}

\begin{longtable}[]{llllll}
\toprule
Version & Status & Priority & Verification Type & Owner
\\\midrule
1 & Approved & Normal &
Test & Fritz Mueller
\\\bottomrule
\end{longtable}

\subsubsection{Verification Elements}
    None.

\subsubsection{Test Items}
The objective of this procedure is to establish the growth trend for
average query execution time of\\
full-table-scan queries in the pool of interest, as a function of query
concurrency.~~The test shall\\
be considered passed if the growth rate is sub-linear (ideally, nearly
flat) within the limits of\\
machine resource exhaustion.



\subsubsection{Predecessors}

\subsubsection{Environment Needs}

\paragraph{Software}

\paragraph{Hardware}

\subsubsection{Input Specification}

\subsubsection{Output Specification}

\subsubsection{Test Procedure}
    \begin{longtable}[]{p{1.3cm}p{2cm}p{13cm}}
    %\toprule
    Step & \multicolumn{2}{@{}l}{Description, Input Data and Expected Result} \\ \toprule
    \endhead

            \multirow{3}{*}{ 1 } & Description &
            \begin{minipage}[t]{13cm}{\footnotesize
            Ensure that Qserv instance under test is up to date and that there is no
other concurrent user activity

            \vspace{\dp0}
            } \end{minipage} \\ \cline{2-3}
            & Test Data &
            \begin{minipage}[t]{13cm}{\footnotesize
                No data.
                \vspace{\dp0}
            } \end{minipage} \\ \cline{2-3}
            & Expected Result &
        \\ \midrule

            \multirow{3}{*}{ 2 } & Description &
            \begin{minipage}[t]{13cm}{\footnotesize
            Inspect and modify the \emph{CONCURRENCY} and \emph{TARGET\_RATES}
dictionaries in the run-Queries.py script. Set \emph{CONCURRENCY}
initially to 1 for the query pool of interest, and to 0 for all other
query pools. ~Set \emph{TARGET\_RATES} for the query pool of interest to
the yearly value per table in QSERV-PRF-10-00.

            \vspace{\dp0}
            } \end{minipage} \\ \cline{2-3}
            & Test Data &
            \begin{minipage}[t]{13cm}{\footnotesize
                No data.
                \vspace{\dp0}
            } \end{minipage} \\ \cline{2-3}
            & Expected Result &
        \\ \midrule

            \multirow{3}{*}{ 3 } & Description &
            \begin{minipage}[t]{13cm}{\footnotesize
            Execute the runQueries.py script and let it run for at least one, but
preferably several, query cycles.

            \vspace{\dp0}
            } \end{minipage} \\ \cline{2-3}
            & Test Data &
            \begin{minipage}[t]{13cm}{\footnotesize
                No data.
                \vspace{\dp0}
            } \end{minipage} \\ \cline{2-3}
            & Expected Result &
        \\ \midrule

            \multirow{3}{*}{ 4 } & Description &
            \begin{minipage}[t]{13cm}{\footnotesize
            Examine log file output and compile performance statistics to obtain a
growth curve point for the pool of interest for the test report.

            \vspace{\dp0}
            } \end{minipage} \\ \cline{2-3}
            & Test Data &
            \begin{minipage}[t]{13cm}{\footnotesize
                No data.
                \vspace{\dp0}
            } \end{minipage} \\ \cline{2-3}
            & Expected Result &
        \\ \midrule

            \multirow{3}{*}{ 5 } & Description &
            \begin{minipage}[t]{13cm}{\footnotesize
            Adjust the \emph{CONCURRENCY} value for the pool of interest and repeat
from the previous step to establish the growth trend and machine
resource exhaustion cutoff for the query pool of interest to an
acceptable degree of accuracy.

            \vspace{\dp0}
            } \end{minipage} \\ \cline{2-3}
            & Test Data &
            \begin{minipage}[t]{13cm}{\footnotesize
                No data.
                \vspace{\dp0}
            } \end{minipage} \\ \cline{2-3}
            & Expected Result &
        \\ \midrule
    \end{longtable}

\subsection{\href{https://jira.lsstcorp.org/secure/Tests.jspa\#/testCase/LVV-T1019}{LVV-T1019}
    - QSERV-PRF-20-05: Source Shared Scan Scaling}\label{lvv-t1019}

\begin{longtable}[]{llllll}
\toprule
Version & Status & Priority & Verification Type & Owner
\\\midrule
1 & Approved & Normal &
Test & Fritz Mueller
\\\bottomrule
\end{longtable}

\subsubsection{Verification Elements}
\begin{itemize}
\item \href{https://jira.lsstcorp.org/browse/LVV-188}{LVV-188} - DMS-REQ-0357-V-01: Result latency for high-volume full-sky queries on
the Object table

\end{itemize}

\subsubsection{Test Items}
This test will show that average completion-time of full-scan queries of
the Source catalog table grow sub-linearly with respect to the number of
simultaneously active full-scan queries, within the limits of machine
resource exhaustion.



\subsubsection{Predecessors}

\subsubsection{Environment Needs}

\paragraph{Software}

\paragraph{Hardware}

\subsubsection{Input Specification}
\begin{itemize}
\tightlist
\item
  A test catalog of appropriate size (see schedule detail in
  QSERV-PRF-10), prepared and ingested into the
  \textbackslash{}product\{\} instance under test as detailed in
  LVV-T1017.
\item
  The concurrency load execution script, runQueries.py, maintained in
  the LSST QSERV github repository here:
  https://github.com/lsst/qserv/blob/master/admin/tools/docker/deployment/in2p3/runQueries.py
\end{itemize}


\subsubsection{Output Specification}
Log files as generated by the runQueries.py test script.


\subsubsection{Test Procedure}
    \begin{longtable}[]{p{1.3cm}p{2cm}p{13cm}}
    %\toprule
    Step & \multicolumn{2}{@{}l}{Description, Input Data and Expected Result} \\ \toprule
    \endhead

                \multirow{3}{*}{\parbox{1.3cm}{ 1-1
                {\scriptsize from \hyperref[lvv-t1018]
                {LVV-T1018} } } }

                & {\small Description} &
                \begin{minipage}[t]{13cm}{\scriptsize
                Ensure that Qserv instance under test is up to date and that there is no
other concurrent user activity

                \vspace{\dp0}
                } \end{minipage} \\ \cdashline{2-3}
                & {\small Test Data} &
                \begin{minipage}[t]{13cm}{\scriptsize
                } \end{minipage} \\ \cdashline{2-3}
                & {\small Expected Result} &
                \\ \hdashline


                \multirow{3}{*}{\parbox{1.3cm}{ 1-2
                {\scriptsize from \hyperref[lvv-t1018]
                {LVV-T1018} } } }

                & {\small Description} &
                \begin{minipage}[t]{13cm}{\scriptsize
                Inspect and modify the \emph{CONCURRENCY} and \emph{TARGET\_RATES}
dictionaries in the run-Queries.py script. Set \emph{CONCURRENCY}
initially to 1 for the query pool of interest, and to 0 for all other
query pools. ~Set \emph{TARGET\_RATES} for the query pool of interest to
the yearly value per table in QSERV-PRF-10-00.

                \vspace{\dp0}
                } \end{minipage} \\ \cdashline{2-3}
                & {\small Test Data} &
                \begin{minipage}[t]{13cm}{\scriptsize
                } \end{minipage} \\ \cdashline{2-3}
                & {\small Expected Result} &
                \\ \hdashline


                \multirow{3}{*}{\parbox{1.3cm}{ 1-3
                {\scriptsize from \hyperref[lvv-t1018]
                {LVV-T1018} } } }

                & {\small Description} &
                \begin{minipage}[t]{13cm}{\scriptsize
                Execute the runQueries.py script and let it run for at least one, but
preferably several, query cycles.

                \vspace{\dp0}
                } \end{minipage} \\ \cdashline{2-3}
                & {\small Test Data} &
                \begin{minipage}[t]{13cm}{\scriptsize
                } \end{minipage} \\ \cdashline{2-3}
                & {\small Expected Result} &
                \\ \hdashline


                \multirow{3}{*}{\parbox{1.3cm}{ 1-4
                {\scriptsize from \hyperref[lvv-t1018]
                {LVV-T1018} } } }

                & {\small Description} &
                \begin{minipage}[t]{13cm}{\scriptsize
                Examine log file output and compile performance statistics to obtain a
growth curve point for the pool of interest for the test report.

                \vspace{\dp0}
                } \end{minipage} \\ \cdashline{2-3}
                & {\small Test Data} &
                \begin{minipage}[t]{13cm}{\scriptsize
                } \end{minipage} \\ \cdashline{2-3}
                & {\small Expected Result} &
                \\ \hdashline


                \multirow{3}{*}{\parbox{1.3cm}{ 1-5
                {\scriptsize from \hyperref[lvv-t1018]
                {LVV-T1018} } } }

                & {\small Description} &
                \begin{minipage}[t]{13cm}{\scriptsize
                Adjust the \emph{CONCURRENCY} value for the pool of interest and repeat
from the previous step to establish the growth trend and machine
resource exhaustion cutoff for the query pool of interest to an
acceptable degree of accuracy.

                \vspace{\dp0}
                } \end{minipage} \\ \cdashline{2-3}
                & {\small Test Data} &
                \begin{minipage}[t]{13cm}{\scriptsize
                } \end{minipage} \\ \cdashline{2-3}
                & {\small Expected Result} &
                \\ \hdashline


        \\ \midrule
    \end{longtable}

\subsection{\href{https://jira.lsstcorp.org/secure/Tests.jspa\#/testCase/LVV-T1020}{LVV-T1020}
    - QSERV-PRF-20-10: Object Source Join Shared Scan Scaling}\label{lvv-t1020}

\begin{longtable}[]{llllll}
\toprule
Version & Status & Priority & Verification Type & Owner
\\\midrule
1 & Approved & Normal &
Test & Fritz Mueller
\\\bottomrule
\end{longtable}

\subsubsection{Verification Elements}
\begin{itemize}
\item \href{https://jira.lsstcorp.org/browse/LVV-188}{LVV-188} - DMS-REQ-0357-V-01: Result latency for high-volume full-sky queries on
the Object table

\end{itemize}

\subsubsection{Test Items}
This test will show that average completion-time of full-scan queries
which join the Object and Source catalog tables grow sub-linearly with
respect to the number of simultaneously active full-scan queries, within
the limits of machine resource exhaustion.



\subsubsection{Predecessors}

\subsubsection{Environment Needs}

\paragraph{Software}

\paragraph{Hardware}

\subsubsection{Input Specification}
\begin{itemize}
\tightlist
\item
  A test catalog of appropriate size (see schedule detail in
  QSERV-PRF-10), prepared and ingested into the
  \textbackslash{}product\{\} instance under test as detailed in
  LVV-T1017.
\item
  The concurrency load execution script, runQueries.py, maintained in
  the LSST QSERV github repository here:
  https://github.com/lsst/qserv/blob/master/admin/tools/docker/deployment/in2p3/runQueries.py
\end{itemize}


\subsubsection{Output Specification}
Log files as generated by the runQueries.py test script


\subsubsection{Test Procedure}
    \begin{longtable}[]{p{1.3cm}p{2cm}p{13cm}}
    %\toprule
    Step & \multicolumn{2}{@{}l}{Description, Input Data and Expected Result} \\ \toprule
    \endhead

                \multirow{3}{*}{\parbox{1.3cm}{ 1-1
                {\scriptsize from \hyperref[lvv-t1018]
                {LVV-T1018} } } }

                & {\small Description} &
                \begin{minipage}[t]{13cm}{\scriptsize
                Ensure that Qserv instance under test is up to date and that there is no
other concurrent user activity

                \vspace{\dp0}
                } \end{minipage} \\ \cdashline{2-3}
                & {\small Test Data} &
                \begin{minipage}[t]{13cm}{\scriptsize
                } \end{minipage} \\ \cdashline{2-3}
                & {\small Expected Result} &
                \\ \hdashline


                \multirow{3}{*}{\parbox{1.3cm}{ 1-2
                {\scriptsize from \hyperref[lvv-t1018]
                {LVV-T1018} } } }

                & {\small Description} &
                \begin{minipage}[t]{13cm}{\scriptsize
                Inspect and modify the \emph{CONCURRENCY} and \emph{TARGET\_RATES}
dictionaries in the run-Queries.py script. Set \emph{CONCURRENCY}
initially to 1 for the query pool of interest, and to 0 for all other
query pools. ~Set \emph{TARGET\_RATES} for the query pool of interest to
the yearly value per table in QSERV-PRF-10-00.

                \vspace{\dp0}
                } \end{minipage} \\ \cdashline{2-3}
                & {\small Test Data} &
                \begin{minipage}[t]{13cm}{\scriptsize
                } \end{minipage} \\ \cdashline{2-3}
                & {\small Expected Result} &
                \\ \hdashline


                \multirow{3}{*}{\parbox{1.3cm}{ 1-3
                {\scriptsize from \hyperref[lvv-t1018]
                {LVV-T1018} } } }

                & {\small Description} &
                \begin{minipage}[t]{13cm}{\scriptsize
                Execute the runQueries.py script and let it run for at least one, but
preferably several, query cycles.

                \vspace{\dp0}
                } \end{minipage} \\ \cdashline{2-3}
                & {\small Test Data} &
                \begin{minipage}[t]{13cm}{\scriptsize
                } \end{minipage} \\ \cdashline{2-3}
                & {\small Expected Result} &
                \\ \hdashline


                \multirow{3}{*}{\parbox{1.3cm}{ 1-4
                {\scriptsize from \hyperref[lvv-t1018]
                {LVV-T1018} } } }

                & {\small Description} &
                \begin{minipage}[t]{13cm}{\scriptsize
                Examine log file output and compile performance statistics to obtain a
growth curve point for the pool of interest for the test report.

                \vspace{\dp0}
                } \end{minipage} \\ \cdashline{2-3}
                & {\small Test Data} &
                \begin{minipage}[t]{13cm}{\scriptsize
                } \end{minipage} \\ \cdashline{2-3}
                & {\small Expected Result} &
                \\ \hdashline


                \multirow{3}{*}{\parbox{1.3cm}{ 1-5
                {\scriptsize from \hyperref[lvv-t1018]
                {LVV-T1018} } } }

                & {\small Description} &
                \begin{minipage}[t]{13cm}{\scriptsize
                Adjust the \emph{CONCURRENCY} value for the pool of interest and repeat
from the previous step to establish the growth trend and machine
resource exhaustion cutoff for the query pool of interest to an
acceptable degree of accuracy.

                \vspace{\dp0}
                } \end{minipage} \\ \cdashline{2-3}
                & {\small Test Data} &
                \begin{minipage}[t]{13cm}{\scriptsize
                } \end{minipage} \\ \cdashline{2-3}
                & {\small Expected Result} &
                \\ \hdashline


        \\ \midrule
    \end{longtable}

\subsection{\href{https://jira.lsstcorp.org/secure/Tests.jspa\#/testCase/LVV-T1021}{LVV-T1021}
    - QSERV-PRF-20-15: Object ForcedSource Join Shared Scan Scaling}\label{lvv-t1021}

\begin{longtable}[]{llllll}
\toprule
Version & Status & Priority & Verification Type & Owner
\\\midrule
1 & Approved & Normal &
Test & Fritz Mueller
\\\bottomrule
\end{longtable}

\subsubsection{Verification Elements}
\begin{itemize}
\item \href{https://jira.lsstcorp.org/browse/LVV-188}{LVV-188} - DMS-REQ-0357-V-01: Result latency for high-volume full-sky queries on
the Object table

\end{itemize}

\subsubsection{Test Items}
This test will show that average completion-time of full-scan queries
which join the Object and ForcedSource catalog tables grow sub-linearly
with respect to the number of simultaneously active full-scan queries,
within the limits of machine resource exhaustion.



\subsubsection{Predecessors}

\subsubsection{Environment Needs}

\paragraph{Software}

\paragraph{Hardware}

\subsubsection{Input Specification}
\begin{itemize}
\tightlist
\item
  A test catalog of appropriate size (see schedule detail in
  QSERV-PRF-10), prepared and ingested into the
  \textbackslash{}product\{\} instance under test as detailed in
  LVV-T1017.
\item
  The concurrency load execution script, runQueries.py, maintained in
  the LSST QSERV github repository here:
  https://github.com/lsst/qserv/blob/master/admin/tools/docker/deployment/in2p3/runQueries.py
\end{itemize}


\subsubsection{Output Specification}
Log files as generated by the runQueries.py test script.


\subsubsection{Test Procedure}
    \begin{longtable}[]{p{1.3cm}p{2cm}p{13cm}}
    %\toprule
    Step & \multicolumn{2}{@{}l}{Description, Input Data and Expected Result} \\ \toprule
    \endhead

                \multirow{3}{*}{\parbox{1.3cm}{ 1-1
                {\scriptsize from \hyperref[lvv-t1018]
                {LVV-T1018} } } }

                & {\small Description} &
                \begin{minipage}[t]{13cm}{\scriptsize
                Ensure that Qserv instance under test is up to date and that there is no
other concurrent user activity

                \vspace{\dp0}
                } \end{minipage} \\ \cdashline{2-3}
                & {\small Test Data} &
                \begin{minipage}[t]{13cm}{\scriptsize
                } \end{minipage} \\ \cdashline{2-3}
                & {\small Expected Result} &
                \\ \hdashline


                \multirow{3}{*}{\parbox{1.3cm}{ 1-2
                {\scriptsize from \hyperref[lvv-t1018]
                {LVV-T1018} } } }

                & {\small Description} &
                \begin{minipage}[t]{13cm}{\scriptsize
                Inspect and modify the \emph{CONCURRENCY} and \emph{TARGET\_RATES}
dictionaries in the run-Queries.py script. Set \emph{CONCURRENCY}
initially to 1 for the query pool of interest, and to 0 for all other
query pools. ~Set \emph{TARGET\_RATES} for the query pool of interest to
the yearly value per table in QSERV-PRF-10-00.

                \vspace{\dp0}
                } \end{minipage} \\ \cdashline{2-3}
                & {\small Test Data} &
                \begin{minipage}[t]{13cm}{\scriptsize
                } \end{minipage} \\ \cdashline{2-3}
                & {\small Expected Result} &
                \\ \hdashline


                \multirow{3}{*}{\parbox{1.3cm}{ 1-3
                {\scriptsize from \hyperref[lvv-t1018]
                {LVV-T1018} } } }

                & {\small Description} &
                \begin{minipage}[t]{13cm}{\scriptsize
                Execute the runQueries.py script and let it run for at least one, but
preferably several, query cycles.

                \vspace{\dp0}
                } \end{minipage} \\ \cdashline{2-3}
                & {\small Test Data} &
                \begin{minipage}[t]{13cm}{\scriptsize
                } \end{minipage} \\ \cdashline{2-3}
                & {\small Expected Result} &
                \\ \hdashline


                \multirow{3}{*}{\parbox{1.3cm}{ 1-4
                {\scriptsize from \hyperref[lvv-t1018]
                {LVV-T1018} } } }

                & {\small Description} &
                \begin{minipage}[t]{13cm}{\scriptsize
                Examine log file output and compile performance statistics to obtain a
growth curve point for the pool of interest for the test report.

                \vspace{\dp0}
                } \end{minipage} \\ \cdashline{2-3}
                & {\small Test Data} &
                \begin{minipage}[t]{13cm}{\scriptsize
                } \end{minipage} \\ \cdashline{2-3}
                & {\small Expected Result} &
                \\ \hdashline


                \multirow{3}{*}{\parbox{1.3cm}{ 1-5
                {\scriptsize from \hyperref[lvv-t1018]
                {LVV-T1018} } } }

                & {\small Description} &
                \begin{minipage}[t]{13cm}{\scriptsize
                Adjust the \emph{CONCURRENCY} value for the pool of interest and repeat
from the previous step to establish the growth trend and machine
resource exhaustion cutoff for the query pool of interest to an
acceptable degree of accuracy.

                \vspace{\dp0}
                } \end{minipage} \\ \cdashline{2-3}
                & {\small Test Data} &
                \begin{minipage}[t]{13cm}{\scriptsize
                } \end{minipage} \\ \cdashline{2-3}
                & {\small Expected Result} &
                \\ \hdashline


        \\ \midrule
    \end{longtable}

\subsection{\href{https://jira.lsstcorp.org/secure/Tests.jspa\#/testCase/LVV-T1022}{LVV-T1022}
    - QSERV-PRF-10-00: Concurrent Query Performance}\label{lvv-t1022}

\begin{longtable}[]{llllll}
\toprule
Version & Status & Priority & Verification Type & Owner
\\\midrule
1 & Approved & Normal &
Test & Fritz Mueller
\\\bottomrule
\end{longtable}

\subsubsection{Verification Elements}
\begin{itemize}
\item \href{https://jira.lsstcorp.org/browse/LVV-188}{LVV-188} - DMS-REQ-0357-V-01: Result latency for high-volume full-sky queries on
the Object table

\item \href{https://jira.lsstcorp.org/browse/LVV-187}{LVV-187} - DMS-REQ-0356-V-01: Radius for low-volume query

\end{itemize}

\subsubsection{Test Items}
This test will check that QSERV is able to meet average query completion
time targets per query class\\
under a representative load of simultaneous high and low volume queries
while running against an appropriately\\
scaled test catalog.



\subsubsection{Predecessors}
LVV-T1016, LVV-T1019, LVV-T1020, LVV-T1021


\subsubsection{Environment Needs}

\paragraph{Software}

\paragraph{Hardware}

\subsubsection{Input Specification}
\begin{itemize}
\tightlist
\item
  A test catalog of appropriate size (see schedule detail in
  QSERV-PRF-10), prepared and ingested into the
  \textbackslash{}product\{\} instance under test as detailed in
  LVV-T1017.
\item
  The concurrency load execution script, runQueries.py, maintained in
  the LSST QSERV github repository here:
  https://github.com/lsst/qserv/blob/master/admin/tools/docker/deployment/in2p3/runQueries.py
\end{itemize}


\subsubsection{Output Specification}
Log files as generated by the runQueries.py test script


\subsubsection{Test Procedure}
    \begin{longtable}[]{p{1.3cm}p{2cm}p{13cm}}
    %\toprule
    Step & \multicolumn{2}{@{}l}{Description, Input Data and Expected Result} \\ \toprule
    \endhead

            \multirow{3}{*}{ 1 } & Description &
            \begin{minipage}[t]{13cm}{\footnotesize
            Inspect and possibly modify the \emph{CONCURRENCY} and
\emph{TARGET\_RATES} dictionaries in the \emph{runQueries.py} script to
adjust the concurrency mix and target execution times per query class.
~Query mixes and target times are to be adjusted per the following
schedule:\\
~\\
\textbf{LV Query Class~}(\#queries, \#time sec):~

\begin{itemize}
\tightlist
\item
  2015(50, 10) 2016(60, 10) 2017(70, 10) 2018(80, 10) 2019(90, 10)
  2020(100, 10)
\end{itemize}

\textbf{FTSObj Query Class~}(\#queries, \#time hours):~

\begin{itemize}
\tightlist
\item
  2015(3, 12) 2016(4, 1) 2017(8, 1) 2018(12, 1) 2019(16, 1) 2020(20, 1)
\end{itemize}

\textbf{FTSSrc Query Class~}(\#queries, \#time hours):~

\begin{itemize}
\tightlist
\item
  2015(1, 12) 2016(1, 12) 2017(2, 12) 2018(3, 12) 2019(4, 12) 2020(5,
  12)
\end{itemize}

\textbf{FTSFSrc Query Class~}(\#queries, \#time hours):~

\begin{itemize}
\tightlist
\item
  2015(0, 0) 2016(1, 12) 2017(2, 12) 2018(3, 12) 2019(4, 12) 2020(5, 12)
\end{itemize}

\textbf{joinObjSrc Query Class~}(\#queries, \#time hours):~

\begin{itemize}
\tightlist
\item
  2015(1, 12) 2016(2, 12) 2017(4, 12) 2018(6, 12) 2019(8, 12) 2020(10,
  12)
\end{itemize}

\textbf{joinObjFSrc Query Class~}(\#queries, \#time hours):~

\begin{itemize}
\tightlist
\item
  2015(0, 0) 2016(1, 12) 2017(2, 12) 2018(3, 12) 2019(4, 12) 2020(5, 12)
\end{itemize}

\textbf{nearN Query Class~}(\#queries, \#time hours):~

\begin{itemize}
\tightlist
\item
  2015(0, 0) 2016(1, 1) 2017(2, 1) 2018(3, 1) 2019(4, 1) 2020(5, 1)
\end{itemize}

            \vspace{\dp0}
            } \end{minipage} \\ \cline{2-3}
            & Test Data &
            \begin{minipage}[t]{13cm}{\footnotesize
                No data.
                \vspace{\dp0}
            } \end{minipage} \\ \cline{2-3}
            & Expected Result &
        \\ \midrule

            \multirow{3}{*}{ 2 } & Description &
            \begin{minipage}[t]{13cm}{\footnotesize
            Ensure that QSERV instance under test is up to date and that there is no
other concurrent user activity.\\
~\\

            \vspace{\dp0}
            } \end{minipage} \\ \cline{2-3}
            & Test Data &
            \begin{minipage}[t]{13cm}{\footnotesize
                No data.
                \vspace{\dp0}
            } \end{minipage} \\ \cline{2-3}
            & Expected Result &
        \\ \midrule

            \multirow{3}{*}{ 3 } & Description &
            \begin{minipage}[t]{13cm}{\footnotesize
            Execute the runQueries.py script and let it run for at least 24hrs.

            \vspace{\dp0}
            } \end{minipage} \\ \cline{2-3}
            & Test Data &
            \begin{minipage}[t]{13cm}{\footnotesize
                No data.
                \vspace{\dp0}
            } \end{minipage} \\ \cline{2-3}
            & Expected Result &
        \\ \midrule

            \multirow{3}{*}{ 4 } & Description &
            \begin{minipage}[t]{13cm}{\footnotesize
            Examine log file output and compile performance statistics for the test
report

            \vspace{\dp0}
            } \end{minipage} \\ \cline{2-3}
            & Test Data &
            \begin{minipage}[t]{13cm}{\footnotesize
                No data.
                \vspace{\dp0}
            } \end{minipage} \\ \cline{2-3}
            & Expected Result &
        \\ \midrule
    \end{longtable}

\subsection{\href{https://jira.lsstcorp.org/secure/Tests.jspa\#/testCase/LVV-T1085}{LVV-T1085}
    - Short Queries Test}\label{lvv-t1085}

\begin{longtable}[]{llllll}
\toprule
Version & Status & Priority & Verification Type & Owner
\\\midrule
1 & Draft & Normal &
Test & Fritz Mueller
\\\bottomrule
\end{longtable}

\subsubsection{Verification Elements}
\begin{itemize}
\item \href{https://jira.lsstcorp.org/browse/LVV-33}{LVV-33} - DMS-REQ-0075-V-01: Catalog Queries

\end{itemize}

\subsubsection{Test Items}
The objective of this test is to ensure that the short queries are
performing as expected



\subsubsection{Predecessors}

\subsubsection{Environment Needs}

\paragraph{Software}

\paragraph{Hardware}

\subsubsection{Input Specification}
QSERV has been set-up following procedure at ~LVV-T1017


\subsubsection{Output Specification}

\subsubsection{Test Procedure}
    \begin{longtable}[]{p{1.3cm}p{2cm}p{13cm}}
    %\toprule
    Step & \multicolumn{2}{@{}l}{Description, Input Data and Expected Result} \\ \toprule
    \endhead

            \multirow{3}{*}{ 1 } & Description &
            \begin{minipage}[t]{13cm}{\footnotesize
            Execute single object selection:\\
~\\
\textbf{SELECT} * \textbf{FROM} Object \textbf{WHERE} objectid =
\textless{}objId\textgreater{}\\
~\\
and record execution time.

            \vspace{\dp0}
            } \end{minipage} \\ \cline{2-3}
            & Test Data &
            \begin{minipage}[t]{13cm}{\footnotesize
                No data.
                \vspace{\dp0}
            } \end{minipage} \\ \cline{2-3}
            & Expected Result &
                \begin{minipage}[t]{13cm}{\footnotesize
                Query runs in less than 1 second.

                \vspace{\dp0}
                } \end{minipage}
        \\ \midrule

            \multirow{3}{*}{ 2 } & Description &
            \begin{minipage}[t]{13cm}{\footnotesize
            Execute spatial area selection from Object:\\
~\\
\textbf{SELECT} * \textbf{FROM} Object \textbf{WHERE}~\\

~ ~ ~ ~ ~ ~ ~ ~ ~ ~ ~ ~ ~ ~ ~qserv\_areaspec\_box(316.582327, −6.839078,
316.653938, −6.781822)

and record execution time.

            \vspace{\dp0}
            } \end{minipage} \\ \cline{2-3}
            & Test Data &
            \begin{minipage}[t]{13cm}{\footnotesize
                No data.
                \vspace{\dp0}
            } \end{minipage} \\ \cline{2-3}
            & Expected Result &
                \begin{minipage}[t]{13cm}{\footnotesize
                Query runs in less than 1 second.

                \vspace{\dp0}
                } \end{minipage}
        \\ \midrule
    \end{longtable}

\subsection{\href{https://jira.lsstcorp.org/secure/Tests.jspa\#/testCase/LVV-T1086}{LVV-T1086}
    - Full table scans, single query at a time}\label{lvv-t1086}

\begin{longtable}[]{llllll}
\toprule
Version & Status & Priority & Verification Type & Owner
\\\midrule
1 & Draft & Normal &
Test & Fritz Mueller
\\\bottomrule
\end{longtable}

\subsubsection{Verification Elements}
\begin{itemize}
\item \href{https://jira.lsstcorp.org/browse/LVV-33}{LVV-33} - DMS-REQ-0075-V-01: Catalog Queries

\end{itemize}

\subsubsection{Test Items}
Full table scans, single query at a time



\subsubsection{Predecessors}

\subsubsection{Environment Needs}

\paragraph{Software}

\paragraph{Hardware}

\subsubsection{Input Specification}
QSERV has been set-up following procedure at ~LVV-T1017


\subsubsection{Output Specification}

\subsubsection{Test Procedure}
    \begin{longtable}[]{p{1.3cm}p{2cm}p{13cm}}
    %\toprule
    Step & \multicolumn{2}{@{}l}{Description, Input Data and Expected Result} \\ \toprule
    \endhead

            \multirow{3}{*}{ 1 } & Description &
            \begin{minipage}[t]{13cm}{\footnotesize
            Execute query:\\
~\\
\textbf{SELECT} ra , decl , u\_psfFlux , g\_psfFlux , r\_psfFlux
\textbf{FROM} Object\\
\textbf{WHERE} y\_shapeIxx \textbf{BETWEEN} 20 \textbf{AND} 20.1\\
~\\
~\\
and record execution time and output size.

            \vspace{\dp0}
            } \end{minipage} \\ \cline{2-3}
            & Test Data &
            \begin{minipage}[t]{13cm}{\footnotesize
                No data.
                \vspace{\dp0}
            } \end{minipage} \\ \cline{2-3}
            & Expected Result &
                \begin{minipage}[t]{13cm}{\footnotesize
                Query expected to run in 20 minutes, output expected less then 100MB

                \vspace{\dp0}
                } \end{minipage}
        \\ \midrule

            \multirow{3}{*}{ 2 } & Description &
            \begin{minipage}[t]{13cm}{\footnotesize
            Execute query:\\
~\\
\textbf{SELECT} COUNT(*) \textbf{FROM} Source \textbf{WHERE} flux
\textbackslash{}\_sinc \textbf{BETWEEN} 1 \textbf{AND} 1.1\\
~\\
and record the execution time

            \vspace{\dp0}
            } \end{minipage} \\ \cline{2-3}
            & Test Data &
            \begin{minipage}[t]{13cm}{\footnotesize
                No data.
                \vspace{\dp0}
            } \end{minipage} \\ \cline{2-3}
            & Expected Result &
                \begin{minipage}[t]{13cm}{\footnotesize
                Query expected to run in 100 minutes

                \vspace{\dp0}
                } \end{minipage}
        \\ \midrule

            \multirow{3}{*}{ 3 } & Description &
            \begin{minipage}[t]{13cm}{\footnotesize
            Execute query:\\
~\\
\textbf{SELECT} COUNT(*) \textbf{FROM} ForcedSource \textbf{WHERE}
psfFlux \textbf{BETWEEN} 0.1 \textbf{AND} 0.2\\
~\\
and record the execution time

            \vspace{\dp0}
            } \end{minipage} \\ \cline{2-3}
            & Test Data &
            \begin{minipage}[t]{13cm}{\footnotesize
                No data.
                \vspace{\dp0}
            } \end{minipage} \\ \cline{2-3}
            & Expected Result &
                \begin{minipage}[t]{13cm}{\footnotesize
                Query expected to run in 50 minutes

                \vspace{\dp0}
                } \end{minipage}
        \\ \midrule
    \end{longtable}

\subsection{\href{https://jira.lsstcorp.org/secure/Tests.jspa\#/testCase/LVV-T1087}{LVV-T1087}
    - Full table joins, single query at a time}\label{lvv-t1087}

\begin{longtable}[]{llllll}
\toprule
Version & Status & Priority & Verification Type & Owner
\\\midrule
1 & Draft & Normal &
Test & Fritz Mueller
\\\bottomrule
\end{longtable}

\subsubsection{Verification Elements}
\begin{itemize}
\item \href{https://jira.lsstcorp.org/browse/LVV-33}{LVV-33} - DMS-REQ-0075-V-01: Catalog Queries

\end{itemize}

\subsubsection{Test Items}
TBC



\subsubsection{Predecessors}

\subsubsection{Environment Needs}

\paragraph{Software}

\paragraph{Hardware}

\subsubsection{Input Specification}
QSERV has been set-up following procedure at ~LVV-T1017


\subsubsection{Output Specification}

\subsubsection{Test Procedure}
    \begin{longtable}[]{p{1.3cm}p{2cm}p{13cm}}
    %\toprule
    Step & \multicolumn{2}{@{}l}{Description, Input Data and Expected Result} \\ \toprule
    \endhead

            \multirow{3}{*}{ 1 } & Description &
            \begin{minipage}[t]{13cm}{\footnotesize
            Execute query:\\
~\\
\textbf{SELECT} o.deepSourceId, s.objectId, s.id, o.ra, o.decl\\
\textbf{~ ~ FROM} Object o, Source s WHERE o.deepSourceId=s.objectId\\
\hspace*{0.333em} ~ \textbf{AND} s . flux\_sinc \textbf{BETWEEN} 0.3
\textbf{AND} 0.31\\
~\\
and record execution time.

            \vspace{\dp0}
            } \end{minipage} \\ \cline{2-3}
            & Test Data &
            \begin{minipage}[t]{13cm}{\footnotesize
                No data.
                \vspace{\dp0}
            } \end{minipage} \\ \cline{2-3}
            & Expected Result &
                \begin{minipage}[t]{13cm}{\footnotesize
                Expected to run as follows:

\begin{itemize}
\tightlist
\item
  2015: less than XXX
\item
  2016:~
\item
  2017:
\item
  2018:
\item
  2019:
\item
  2020:
\end{itemize}

                \vspace{\dp0}
                } \end{minipage}
        \\ \midrule

            \multirow{3}{*}{ 2 } & Description &
            \begin{minipage}[t]{13cm}{\footnotesize
            Execute query:\\
~\\
\textbf{SELECT} o.deepSourceId, f.psfFlux \textbf{FROM} Object o,
ForcedSource f\\
\textbf{~ ~ WHERE} o.deepSourceId=f.deepSourceId\\
\textbf{~ ~ AND} f . psfFlux \textbf{BETWEEN} 0.13 \textbf{AND} 0.14\\
~\\
and record execution time.

            \vspace{\dp0}
            } \end{minipage} \\ \cline{2-3}
            & Test Data &
            \begin{minipage}[t]{13cm}{\footnotesize
                No data.
                \vspace{\dp0}
            } \end{minipage} \\ \cline{2-3}
            & Expected Result &
                \begin{minipage}[t]{13cm}{\footnotesize
                to be specied

                \vspace{\dp0}
                } \end{minipage}
        \\ \midrule
    \end{longtable}

\subsection{\href{https://jira.lsstcorp.org/secure/Tests.jspa\#/testCase/LVV-T1088}{LVV-T1088}
    - Concurrent scans and stress test}\label{lvv-t1088}

\begin{longtable}[]{llllll}
\toprule
Version & Status & Priority & Verification Type & Owner
\\\midrule
1 & Draft & Normal &
Test & Fritz Mueller
\\\bottomrule
\end{longtable}

\subsubsection{Verification Elements}
\begin{itemize}
\item \href{https://jira.lsstcorp.org/browse/LVV-33}{LVV-33} - DMS-REQ-0075-V-01: Catalog Queries

\end{itemize}

\subsubsection{Test Items}
to be completed



\subsubsection{Predecessors}

\subsubsection{Environment Needs}

\paragraph{Software}

\paragraph{Hardware}

\subsubsection{Input Specification}
QSERV has been set-up following procedure at ~LVV-T1017


\subsubsection{Output Specification}

\subsubsection{Test Procedure}
    \begin{longtable}[]{p{1.3cm}p{2cm}p{13cm}}
    %\toprule
    Step & \multicolumn{2}{@{}l}{Description, Input Data and Expected Result} \\ \toprule
    \endhead

            \multirow{3}{*}{ 1 } & Description &
            \begin{minipage}[t]{13cm}{\footnotesize
            \emph{2 Object scans\\
}

            \vspace{\dp0}
            } \end{minipage} \\ \cline{2-3}
            & Test Data &
            \begin{minipage}[t]{13cm}{\footnotesize
                No data.
                \vspace{\dp0}
            } \end{minipage} \\ \cline{2-3}
            & Expected Result &
                \begin{minipage}[t]{13cm}{\footnotesize
                to be defined

                \vspace{\dp0}
                } \end{minipage}
        \\ \midrule

            \multirow{3}{*}{ 2 } & Description &
            \begin{minipage}[t]{13cm}{\footnotesize
            5 Object scans

            \vspace{\dp0}
            } \end{minipage} \\ \cline{2-3}
            & Test Data &
            \begin{minipage}[t]{13cm}{\footnotesize
                No data.
                \vspace{\dp0}
            } \end{minipage} \\ \cline{2-3}
            & Expected Result &
                \begin{minipage}[t]{13cm}{\footnotesize
                to be defined

                \vspace{\dp0}
                } \end{minipage}
        \\ \midrule

            \multirow{3}{*}{ 3 } & Description &
            \begin{minipage}[t]{13cm}{\footnotesize
            60 Object scans

            \vspace{\dp0}
            } \end{minipage} \\ \cline{2-3}
            & Test Data &
            \begin{minipage}[t]{13cm}{\footnotesize
                No data.
                \vspace{\dp0}
            } \end{minipage} \\ \cline{2-3}
            & Expected Result &
                \begin{minipage}[t]{13cm}{\footnotesize
                to be defined

                \vspace{\dp0}
                } \end{minipage}
        \\ \midrule
    \end{longtable}

\subsection{\href{https://jira.lsstcorp.org/secure/Tests.jspa\#/testCase/LVV-T1089}{LVV-T1089}
    - Load test 70 LV + 20 HV}\label{lvv-t1089}

\begin{longtable}[]{llllll}
\toprule
Version & Status & Priority & Verification Type & Owner
\\\midrule
1 & Draft & Normal &
Test & Fritz Mueller
\\\bottomrule
\end{longtable}

\subsubsection{Verification Elements}
\begin{itemize}
\item \href{https://jira.lsstcorp.org/browse/LVV-33}{LVV-33} - DMS-REQ-0075-V-01: Catalog Queries

\end{itemize}

\subsubsection{Test Items}
to be completed



\subsubsection{Predecessors}

\subsubsection{Environment Needs}

\paragraph{Software}

\paragraph{Hardware}

\subsubsection{Input Specification}
QSERV has been set-up following procedure at ~LVV-T1017


\subsubsection{Output Specification}

\subsubsection{Test Procedure}
    \begin{longtable}[]{p{1.3cm}p{2cm}p{13cm}}
    %\toprule
    Step & \multicolumn{2}{@{}l}{Description, Input Data and Expected Result} \\ \toprule
    \endhead

            \multirow{3}{*}{ 1 } & Description &
            \begin{minipage}[t]{13cm}{\footnotesize
            70 low volume and 20 high volume queries (8 scans for Object, 2 scans
for Source, 2 scans for ForcedSource, 4 Object-Source joins, 2
Object-ForcedSource joins and 2 NearNeigh- bor queries), all running
simultaneously with appropriate sleep in between queries to enforce the
mix we are aiming for\\
~\\

            \vspace{\dp0}
            } \end{minipage} \\ \cline{2-3}
            & Test Data &
            \begin{minipage}[t]{13cm}{\footnotesize
                No data.
                \vspace{\dp0}
            } \end{minipage} \\ \cline{2-3}
            & Expected Result &
                \begin{minipage}[t]{13cm}{\footnotesize
                to be defined

                \vspace{\dp0}
                } \end{minipage}
        \\ \midrule
    \end{longtable}

\subsection{\href{https://jira.lsstcorp.org/secure/Tests.jspa\#/testCase/LVV-T1090}{LVV-T1090}
    - Heavy load test 100 LV + 30 HV}\label{lvv-t1090}

\begin{longtable}[]{llllll}
\toprule
Version & Status & Priority & Verification Type & Owner
\\\midrule
1 & Draft & Normal &
Test & Fritz Mueller
\\\bottomrule
\end{longtable}

\subsubsection{Verification Elements}
\begin{itemize}
\item \href{https://jira.lsstcorp.org/browse/LVV-33}{LVV-33} - DMS-REQ-0075-V-01: Catalog Queries

\end{itemize}

\subsubsection{Test Items}
to be completed



\subsubsection{Predecessors}

\subsubsection{Environment Needs}

\paragraph{Software}

\paragraph{Hardware}

\subsubsection{Input Specification}
QSERV has been set-up following procedure at ~LVV-T1017


\subsubsection{Output Specification}

\subsubsection{Test Procedure}
    \begin{longtable}[]{p{1.3cm}p{2cm}p{13cm}}
    %\toprule
    Step & \multicolumn{2}{@{}l}{Description, Input Data and Expected Result} \\ \toprule
    \endhead

            \multirow{3}{*}{ 1 } & Description &
            \begin{minipage}[t]{13cm}{\footnotesize
            100 low volume and 30 high volume queries (12 scans for Object, 3 scans
for Source, 3 scans for ForcedSource, 6 Object-Source joins, 3
Object-ForcedSource join and 3 NearNeigh- bor queries), all running
simultaneously with appropriate sleep in between queries to enforce the
mix we are aiming for

            \vspace{\dp0}
            } \end{minipage} \\ \cline{2-3}
            & Test Data &
            \begin{minipage}[t]{13cm}{\footnotesize
                No data.
                \vspace{\dp0}
            } \end{minipage} \\ \cline{2-3}
            & Expected Result &
                \begin{minipage}[t]{13cm}{\footnotesize
                to be defined

                \vspace{\dp0}
                } \end{minipage}
        \\ \midrule
    \end{longtable}

\appendix


\newpage
\appendix
% generated from JIRA project LVV
% using template at /var/jenkins_home/.local/lib/python3.7/site-packages/docsteady/templates/dm-spec-appendix.latex.jinja2.
% Collecting ATM data from folder: "/Data Management/Software Products/Supporting SW/Distrib Database|LDM-552"
% using dosteady version 1.2rc2
% Please do not edit -- update information in Jira instead

\section{Traceability}
\scriptsize{
\begin{longtable}[]{p{13cm}p{3cm}}
\toprule
Verification Elements & Test Cases\tabularnewline
\midrule
\endhead

\href{https://jira.lsstcorp.org/browse/LVV-33}{LVV-33 - DMS-REQ-0075-V-01: Catalog Queries
}
& {
\hyperref[lvv-t1085]{LVV-T1085}
\hyperref[lvv-t1086]{LVV-T1086}
\hyperref[lvv-t1087]{LVV-T1087}
\hyperref[lvv-t1089]{LVV-T1089}
\hyperref[lvv-t1090]{LVV-T1090}
\hyperref[lvv-t1091]{LVV-T1091}
} \\
\href{https://jira.lsstcorp.org/browse/LVV-185}{LVV-185 - DMS-REQ-0354-V-01: Result latency for high-volume complex queries
}
& {
\hyperref[lvv-t1088]{LVV-T1088}
} \\
\href{https://jira.lsstcorp.org/browse/LVV-188}{LVV-188 - DMS-REQ-0357-V-01: Result latency for high-volume full-sky queries on
the Object table
}
& {
\hyperref[lvv-t1088]{LVV-T1088}
} \\
\href{https://jira.lsstcorp.org/browse/LVV-3403}{LVV-3403 - DMS-REQ-0361-V-01: Simultaneous users for high-volume queries
}
& {
\hyperref[lvv-t1088]{LVV-T1088}
} \\
\tabularnewline
\bottomrule
\end{longtable}
} % end scriptsize


\section{References\label{sect:references}}
\renewcommand{\refname}{}
\bibliography{lsst,refs,books,refs_ads,local.bib}

\section{Acronyms \label{sect:acronyms}} % include acronyms.tex generated by the generateAcronyms.py (in texmf/scripts)
\addtocounter{table}{-1}
\begin{longtable}{|p{0.145\textwidth}|p{0.8\textwidth}|}\hline
\textbf{Acronym} & \textbf{Description}  \\\hline

ATM & Adaptavist Test Management \\\hline
Archive & The repository for documents required by the NSF to be kept. These include documents related to design and development, construction, integration, test, and operations of the LSST observatory system. The archive is maintained using the enterprise content management system DocuShare, which is accessible through a link on the project website www.project.lsst.org. \\\hline
CC & Change Control \\\hline
CI & Cyber Infrastructure \\\hline
Center & An entity managed by AURA that is responsible for execution of a federally funded project \\\hline
Change Control & The systematic approach to managing all changes to the LSST system, including technical data and policy documentation. The purpose is to ensure that no unnecessary changes are made, all changes are documented, and resources are used efficiently and appropriately. \\\hline
DAC & Data Access Center \\\hline
DM & Data Management \\\hline
DMS & Data Management Subsystem \\\hline
DR & Data Release \\\hline
DRP & Data Release Production \\\hline
Data Access Center & Part of the LSST Data Management System, the US and Chilean DACs will provide authorized access to the released LSST data products, software such as the Science Platform, and computational resources for data analysis. The US DAC also includes a service for distributing bulk data on daily and annual (Data Release) timescales to partner institutions, collaborations, and LSST Education and Public Outreach (EPO).  \\\hline
Data Management & The LSST Subsystem responsible for the Data Management System (DMS), which will capture, store, catalog, and serve the LSST dataset to the scientific community and public. The DM team is responsible for the DMS architecture, applications, middleware, infrastructure, algorithms, and Observatory Network Design. DM is a distributed team working at LSST and partner institutions, with the DM Subsystem Manager located at LSST headquarters in Tucson. \\\hline
Data Management Subsystem & The subsystems within Data Management may contain a defined combination of hardware, a software stack, a set of running processes, and the people who manage them: they are a major component of the DM System operations. Examples include the 'Archive Operations Subsystem' and the 'Data Processing Subsystem'"." \\\hline
Data Management System & The computing infrastructure, middleware, and applications that process, store, and enable information extraction from the LSST dataset; the DMS will process peta-scale data volume, convert raw images into a faithful representation of the universe, and archive the results in a useful form. The infrastructure layer consists of the computing, storage, networking hardware, and system software. The middleware layer handles distributed processing, data access, user interface, and system operations services. The applications layer includes the data pipelines and the science data archives' products and services. \\\hline
Data Release & The approximately annual reprocessing of all LSST data, and the installation of the resulting data products in the LSST Data Access Centers, which marks the start of the two-year proprietary period. \\\hline
DocuShare & The trade name for the enterprise management software used by LSST to archive and manage documents \\\hline
Document & Any object (in any application supported by DocuShare or design archives such as PDMWorks or GIT) that supports project management or records milestones and deliverables of the LSST Project \\\hline
EPO & Education and Public Outreach \\\hline
ForcedSource & DRP table resulting from forced photometry. \\\hline
HSC & Hyper Suprime-Cam \\\hline
Handle & The unique identifier assigned to a document uploaded to DocuShare \\\hline
IRSA & Infrared Science Archive \\\hline
LDM & LSST Data Management (Document Handle) \\\hline
LSST & Large Synoptic Survey Telescope \\\hline
NCSA & National Center for Supercomputing Applications \\\hline
Object & In LSST nomenclature this refers to an astronomical object, such as a star, galaxy, or other physical entity. E.g., comets, asteroids are also Objects but typically called a Moving Object or a Solar System Object (SSObject). One of the DRP data products is a table of Objects detected by LSST which can be static, or change brightness or position with time. \\\hline
PDAC & Prototype Data Access Center \\\hline
QA & Quality Assurance \\\hline
Qserv & LSST's distributed parallel database. This database server is used for collecting, storing, and serving LSST Data Release Catalogs and Project metadata, and is part of the Software Stack. \\\hline
RAM & Random Access Memory \\\hline
SDSS & Sloan Digital Sky Survey \\\hline
Science Platform & A set of integrated web applications and services deployed at the LSST Data Access Centers (DACs) through which the scientific community will access, visualize, and perform next-to-the-data analysis of the LSST data products. \\\hline
Scope & The work needed to be accomplished in order to deliver the product, service, or result with the specified features and functions \\\hline
Sloan Digital Sky Survey & is a digital survey of roughly 10,000 square degrees of sky around the north Galactic pole, plus a ~300 square degree stripe along the celestial equator. \\\hline
Software Stack & Often referred to as the LSST Stack, or just The Stack, it is the collection of software written by the LSST Data Management Team to process, generate, and serve LSST images, transient alerts, and catalogs. The Stack includes the LSST Science Pipelines, as well as packages upon which the DM software depends. It is open source and publicly available. \\\hline
Solar System Object & A solar system object is an astrophysical object that is identified as part of the Solar System: planets and their satellites, asteroids, comets, etc. This class of object had historically been referred to within the LSST Project as Moving Objects. \\\hline
Source & A single detection of an astrophysical object in an image, the characteristics for which are stored in the Source Catalog of the DRP database. The association of Sources that are non-moving lead to Objects; the association of moving Sources leads to Solar System Objects. (Note that in non-LSST usage "source" is often used for what LSST calls an Object.) \\\hline
Specification & One or more performance parameter(s) being established by a requirement that the delivered system or subsystem must meet \\\hline
Stripe 82 & A 2.5° wide equatorial band of sky covering roughly 300 square degrees that was observed repeatedly in 5 passbands during the course of the SDSS, In part for calibration purposes. \\\hline
Subsystem & A set of elements comprising a system within the larger LSST system that is responsible for a key technical deliverable of the project. \\\hline
Subsystem Manager & responsible manager for an LSST subsystem; he or she exercises authority, within prescribed limits and under scrutiny of the Project Manager, over the relevant subsystem's cost, schedule, and work plans \\\hline
TB & TeraByte \\\hline
US & United States \\\hline
Validation & A process of confirming that the delivered system will provide its desired functionality; overall, a validation process includes the evaluation, integration, and test activities carried out at the system level to ensure that the final developed system satisfies the intent and performance of that system in operations \\\hline
Verification & The process of evaluating the design, including hardware and software - to ensure the requirements have been met;  verification (of requirements) is performed by test, analysis, inspection, and/or demonstration \\\hline
astronomical object & A star, galaxy, asteroid, or other physical object of astronomical interest. Beware: in non-LSST usage, these are often known as sources. \\\hline
calibration & The process of translating signals produced by a measuring instrument such as a telescope and camera into physical units such as flux, which are used for scientific analysis. Calibration removes most of the contributions to the signal from environmental and instrumental factors, such that only the astronomical component remains. \\\hline
forced photometry & A measurement of the photometric properties of a source, or expected source, with one or more parameters held fixed. Most often this means fixing the location of the center of the brightness profile (which may be known or predicted in advance), and measuring other properties such as total brightness, shape, and orientation. Forced photometry will be done for all Objects in the Data Release Production. \\\hline
metadata & General term for data about data, e.g., attributes of astronomical objects (e.g. images, sources, astroObjects, etc.) that are characteristics of the objects themselves, and facilitate the organization, preservation, and query of data sets. (E.g., a FITS header contains metadata). \\\hline
monitoring & In DM QA, this refers to the process of collecting, storing, aggregating and visualizing metrics. \\\hline
stack & a grouping, usually in layers (hence stack), of software packages and services to achieve a common goal. Often providing a higher level set of end user oriented services and tools \\\hline
\end{longtable}


\end{document}
